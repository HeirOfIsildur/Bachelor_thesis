%%%%%%%%
% Ukazka beamer prezentace se zcela zakladnim nastavenim
%%%%%%%%%%

\documentclass{beamer}


% Nastaveni vzhledu a barev:

% Tema:
\usetheme{Warsaw}
% jine volby -> viz https://deic-web.uab.cat/~iblanes/beamer_gallery/index_by_theme.html

 \usecolortheme{default}
 % jine volby viz  -> https://deic-web.uab.cat/~iblanes/beamer_gallery/index_by_color.html
% pripadne dalsi barvy lze za pouziti BeamerColor.sty


\setbeamertemplate{navigation symbols}{} % pro odstraneni navigacnich symbolu

% pokud chceme trochu jine fonty, pak lze zavolat dodatecne baliky:
%\usepackage{times}
%\usepackage[T1]{fontenc}
% take viz https://deic-web.uab.cat/~iblanes/beamer_gallery/index_by_font.html

% Pokud chceme nazvy tluste:
%   \usefonttheme{structurebold}




%%%%%%%%%%%
% Standardni nastaveni - balicky, definice funkci atd.

\usepackage[czech]{babel}
\usepackage[utf8]{inputenc}
\usepackage{amsmath}
\usepackage[ddmmyyyy]{datetime}
\usepackage{graphicx} % chceme-li vkladat obrazky
\DeclareMathOperator{\pr}{\mathsf{P}}
\usepackage{xparse}
%%% This file contains definitions of various useful macros and environments %%%
%%% Please add more macros here instead of cluttering other files with them. %%%

%%% Minor tweaks of style

% These macros employ a little dirty trick to convince LaTeX to typeset
% chapter headings sanely, without lots of empty space above them.
% Feel free to ignore.
\makeatletter
\def\@makechapterhead#1{
  {\parindent \z@ \raggedright \normalfont
   \Huge\bfseries \thechapter. #1
   \par\nobreak
   \vskip 20\p@
}}
\def\@makeschapterhead#1{
  {\parindent \z@ \raggedright \normalfont
   \Huge\bfseries #1
   \par\nobreak
   \vskip 20\p@
}}
\makeatother

% This macro defines a chapter, which is not numbered, but is included
% in the table of contents.
\def\chapwithtoc#1{
\chapter*{#1}
\addcontentsline{toc}{chapter}{#1}
}

% Draw black "slugs" whenever a line overflows, so that we can spot it easily.
\overfullrule=1mm

%%% Macros for definitions, theorems, claims, examples, ... (requires amsthm package)

\theoremstyle{plain}
\newtheorem{thm}{Theorem}
\newtheorem{lemma}[thm]{Lemma}
\newtheorem{claim}[thm]{Claim}

\theoremstyle{plain}
\newtheorem{defn}{Definition}

\theoremstyle{remark}
\newtheorem*{cor}{Corollary}
\newtheorem*{rem}{Remark}
\newtheorem*{example}{Example}

%%% An environment for proofs

\newenvironment{myproof}{
  \par\medskip\noindent
  \textit{Proof}.
}{
\newline
\rightline{$\qedsymbol$}
}

%%% An environment for typesetting of program code and input/output
%%% of programs. (Requires the fancyvrb package -- fancy verbatim.)

\DefineVerbatimEnvironment{code}{Verbatim}{fontsize=\small, frame=single}

%%% The field of all real and natural numbers
\newcommand{\R}{\mathbb{R}}
\newcommand{\N}{\mathbb{N}}

%%% Useful operators for statistics and probability
\DeclareMathOperator{\pr}{\textsf{P}}
\DeclareMathOperator{\E}{\textsf{E}\,}
\DeclareMathOperator{\var}{\textrm{var}}
\DeclareMathOperator{\sd}{\textrm{sd}}

%%% Transposition of a vector/matrix
\newcommand{\T}[1]{#1^\top}

%%% Various math goodies
\newcommand{\goto}{\rightarrow}
\newcommand{\gotop}{\stackrel{P}{\longrightarrow}}
\newcommand{\maon}[1]{o(n^{#1})}
\newcommand{\abs}[1]{\left|{#1}\right|}
\newcommand{\dint}{\int_0^\tau\!\!\int_0^\tau}
\newcommand{\isqr}[1]{\frac{1}{\sqrt{#1}}}

%%% Various table goodies
\newcommand{\pulrad}[1]{\raisebox{1.5ex}[0pt]{#1}}
%%%%%%%%%%%%%%%\newcommand{\mc}[1]{\multicolumn{1}{c}{#1}}

%%% Moje vlastní makra

\newcommand{\Z}{\mathbb{Z}}
\newcommand{\pinf}{{+\infty}} %%%Plus nekonečno
\newcommand{\lb}{{\left\{}}
\newcommand{\rb}{{\right\}}}
\newcommand{\dosad}{:=}
\NewDocumentCommand {\suma} { O{i} O{1} O{n} } {\sum\limits_{#1=#2}^{#3}}
\newcommand{\indikator}[1]{\mathbf{1}_{[#1]}}
\newcommand{\cen}{\textbf{check english}}
\newcommand{\rw}{$\left(\{S_n\}_{n=0}^\pinf, p \right)$}
\newcommand{\iid}{independent and identically distributed}
\newcommand{\Iid}{independent and identically distributed}
\newcommand{\Lim}[1]{\raisebox{0.5ex}{\scalebox{0.8}{$\displaystyle \lim_{#1}\;$}}}
\newcommand{\Time}{time }
\newcommand{\intime}{in time }
\newcommand{\Wlog}{Without loss of generality}
\newcommand{\h}{\frac{1}{2}}
\newcommand{\ce}{\textbf{check english }}
\newcommand{\Mid}{\;\middle\vert\;}


% Autor, nazev atd

\title[Occupation of a set time of
random walks] {Occupation of a set time of
random walks}
% pokud je nazev prilis dlouhy, muzeme dat do [] kratsi verzi

\author{Jan Janoušek}

\date{\today}



% Hlavni dokument:

\begin{document}

% uvodni slide
\begin{frame}
 \titlepage

{\small
\begin{center}
Supervised by: doc. RNDr. Daniel Hlubinka, Ph.D.
\end{center}
}

\end{frame}

% slide s obsahem prezentace
\begin{frame}{Overview}
 \tableofcontents
\end{frame}


\section{Introduction}



\subsection{Motivation}

\begin{frame}{Motivation}
 \begin{itemize}
\item<1->{How long can you profit if you buy $n$ independent stocks}
\item<2->{Gambler's ruin}
\item<3->{Arcsine laws}
\end{itemize}


\end{frame}


\begin{frame}{Terminology}
\begin{itemize}
\item<1-> Random walk with parameter $p$:  $\left(\{S_n\}_{\n \in \mathbb{N}_0} , p \right)$
\item<2-> Return to the origin: $S_n=0$
\item<3-> Probability of return to origin: $u_n\dosad \pr\left(S_n=0\right)$
\item<4-> First return to the origin: $S_1, S_2, \ldots, S_{n-1}\neq 0, S_{2n}=0$
\item<5-> Probability of the first return to the origin $f_n\dosad \pr\left(S_1, S_2, \ldots, S_{n-1}\neq 0, S_{2n}=0\right)$
\item<6-> Probability of last return to origin at time $k$: $\beta_{n}\left( k\right)$
\end{itemize}
\end{frame}
\subsection{Definition of a random walk}
\begin{frame}{Definition of a random walk}
\begin{block}{Definition}
Let $\{X_n\}_{n=0}^\pinf$ be a sequence of  \iid $\{-1,1\}$-valued random variables, that for some $p\in(0,1)$ and for $n\in \mathbb{N}$ satisfy $\pr \left( X_n=1 \right) =p$ and $\pr \left( X_n=-1 \right) =1-p=:q$.

Let $S_0=0$ and $S_n=\suma[i][1][n]X_i$. We call the pair $ \left( \{S_n\}_{n=0}^\pinf, p \right) $ \emph{Simple~random walk in $\mathbb{Z}$}.
\end{block}
\begin{itemize}
\item<2->Three possible outcomes:
\begin{enumerate}
\item<3-> $p = q$ the walk stays approximately around origin (symmetric)
\item<4-> $p > q$ the walk shifts rightwards/upwards for $n \to \pinf$
\item<5-> $p < q$ the walk shifts leftwards/downwards for $n \to \pinf$
\end{enumerate}
\end{itemize}
\end{frame}

\section{Basic properties of random walk}
\subsection{Probability of position at time}
\begin{frame}{Probability of position at time}
\begin{block}{Theorem}
   \Lrw.
\[
\pr \left( S_n=x \right) =
\begin{cases}
\binom{n}{\frac{n+x}{2}}p^{\frac{n+x}{2}}q^{\frac{n-x}{2}} & \text{for $x \in A_n$ },\\
0, & \text{for $x \not\in A_n$},
\end{cases}
\]
where $A_n=\{z \in \Z; \abs{z}\leq n, \frac{z+n}{2}\in \Z \}$
\end{block}
\end{frame}

\begin{frame}{Probability of return to origin}
\begin{block}{Theorem}
   \Lrw.
\[
\pr \left( S_n=0 \right) =
\begin{cases}
\binom{n}{\frac{n}{2}}p^{\frac{n}{2}}q^{\frac{n}{2}} & \text{for $n$ even },\\
0, & \text{for $n$ odd},
\end{cases}
\]
\end{block}
\end{frame}
\subsection{Probability of no return to origin}
\begin{frame}{Probability of no return to origin}
\begin{block}{Theorem}
   \Lrws.
\[
\begin{split}
\pr \left( S_n=0 \right) &=\pr \left( S_1, S_2, \ldots, S_{2n}\neq 0 \right) =\\
& =2\pr \left( S_1, S_2, \ldots, S_{2n}> 0 \right)=\pr \left( S_1, S_2, \ldots, S_{2n}\geq 0 \right)
\end{split}
\]
\end{block}
\end{frame}

\subsection{Arcsine law for last visits}
\begin{frame}{Arcsine law for last visits}

\begin{block}{Theorem}
\Lrws. Then the probability that the last return to the origin up to time $2n$ happened at time $2k$ is
\[
\alpha_{2n} \left( 2k \right)=\pr\left( S_{2k}=0\right)\pr\left( S_{2n-2k}=0\right).
\]
\end{block}

\begin{itemize}
\item<1->\[
   \pr\left( S_{2k}=0\right)=\binom{2k}{k}2^{-2k}
   =\frac{\left( 2k\right)!}{\left(k!\right)^2}2^{-2k}
   \sim \frac{\left( \frac{2k}{e}\right)^{2k}\sqrt{4\pi k}}{\left(\left( \frac{k}{e}\right)^{k}\sqrt{2\pi k}\right)^2}2^{-2k}
\]
\item<2->\[
=\frac{\left( \frac{k}{e}\right)^{2k}2^{2k}2\sqrt{\pi k}}{\left( \frac{k}{e}\right)^{2k}2\pi k}2^{-2k}=\frac{1}{\sqrt{\pi k}}
\]
\end{itemize}
\end{frame}
\begin{frame}{Arcsine law for last visits-continuing}
\begin{itemize}
   \item<1-> following same procedure we get that $\pr\left(S_{2n-2k}=0\right)\sim \frac{1}{\sqrt{\pi \left(n-k\right)}}$
   \item<2-> therefore $\beta_{2n}\left( 2k\right)=\frac{1}{\pi \sqrt{k\left(n-k\right)}}=\frac{1}{\pi \sqrt{\frac{k}{n}\left(1-\frac{k}{n}\right)}}$
   \item<3-> therefore we get the distribution function:
   \[
   F(nx)=\pr(X\leq nx)=\int_0^{nx}\frac{1}{\pi \sqrt{y(n-y)}} dy=\frac{2}{\pi}\arcsin{\sqrt{x}}
   \]
\end{itemize}

\end{frame}

\subsection{Arcsine law for sojourn times}
\begin{frame}{Arcsine law for sojourn times}

\begin{block}{Theorem}
\Lrws. Then the probability that up to time $2n$ the walk spends $2k$ steps on the positive side is
\[
\alpha_{2n} \left( 2k \right)=\pr\left( S_{2n}=0\right)\pr\left( S_{2n-2k}=0\right).
\]
\end{block}
\end{frame}

\section{Multi dimensional random walk}
\renewcommand{\rw}{$\left( \{\boldsymbol{S_n}\}_{n=0}^\pinf, \boldsymbol{p}\right)$}
\subsection{Type I random walk}
\begin{frame}{Type I random walk-definition}
\begin{block}{Definition}
  Let $m \in \N$. Let $e_i$ denote $i$-th vector of standard basis in $\R^m$. Let $\{X_n\}_{n=0}^\pinf$ be a sequence of \iid random variables with values in $\{ -e_1, +e_1, -e_2, +e_2, \ldots, -e_m, +e_m\}$.
 That $\forall n \in \N$ satisfy the conditions $\pr\left(X_n=e_i \right)=p_i \in \left(0,1\right)$, $\pr\left(X_n=-e_i \right)=q_i \in \left(0,1\right)$ and $\suma[i][1][m] p_i+\suma[i][1][m]q_i=1$.
 Let $\boldsymbol{p}=\begin{pmatrix}
  p_1, p_2, \ldots, p_m
\end{pmatrix}^T$, $\boldsymbol{q}=\begin{pmatrix}
  q_1, q_2, \ldots, q_m
\end{pmatrix}^T$ and $\boldsymbol{0}=\begin{pmatrix}
   0, 0, \ldots, 0
 \end{pmatrix}^T$. Let $S_0=\boldsymbol{0}$ and $S_n=\suma X_i$. Then the triplet $\left(\{\boldsymbol{S_n}\}_{n=0}^\pinf, \boldsymbol{p}, \boldsymbol{q} \right)$ is called \emph{Type I random walk in $\mathbb{Z}^m$}.
\end{block}
\end{frame}
\subsection{Type II random walk}
\begin{frame}{Type II random walk-definition}
\begin{block}{Definition}
  Let $m \in \N$. $\forall n \in \N$, let $\boldsymbol{X_n}=\begin{pmatrix} \x[1] & \x[2]& \ldots, \x[m]\end{pmatrix}^T$, where $\{\x[i]\}_{i=1}^{m}$ are $\forall n \in \N$ independent.

 Let $\forall i \in \jm \ \x[i]$ be a $\{-1,+1\}$-valued random variable with probabilities $\pr \left(\x[i]=+1\right)=p_i \in \left(0,1\right)$ and $\pr \left(\x[i]=-1\right)=1-p_i=:q_i \in  \left(0,1 \right) \forall n \in \N.$

 Let $\{X_n\}_{n=0}^\pinf$ be a sequence of \iid random vectors. Let $\boldsymbol{S_0}=\boldsymbol{0}$ and $\forall n \in \N: \boldsymbol{S_n}=\suma \boldsymbol{X_i}$ and $\boldsymbol{p}=\begin{pmatrix}
  p_1, p_2, \ldots, p_m
\end{pmatrix}^T$. Then the pair $\left(\{\mathbf{S_n}\}_{n=0}^\pinf, \boldsymbol{p} \right)$ is called \emph{Type II random walk in $\mathbb{Z}^m$}.
\end{block}
\end{frame}


\begin{frame}{Type II random walk probability of position at time}
\begin{block}{Theorem}
  \Lrwm. Let $\boldsymbol{y}=\begin{pmatrix}
  y_1, y_2, \ldots, y_m
 \end{pmatrix}^T\in \Z^m.$ Then following equation stands:
 \[
 \pr \left( S_n=\boldsymbol{y} \right) =
 \prod_{i=1}^{m}\binom{n}{\frac{y_i+n}{2}}{p_i}^{\frac{n+y_i}{2}}{q_i}^{\frac{n-y_i}{2}}\]
 if $\forall i \in \jm:y_i\in A_n, 0$ otherwise.

\end{block}
\end{frame}

\begin{frame}{Type II random walk probability of position at time}
\begin{block}{Theorem}
  Let $\{\mathbf{\mathbf{S_n}}\}_{n=0}^\pinf$ be a Symmetric type II random walk in $\Z^m$. Let $C$ be a closed orthant in $Z^m$.
  \[ \pr \left(\boldsymbol{S_n}\in C\right)=\left( \pr\left(S_{2n}=0\right)\right)^m=\left( 2^{-2n}\binom{2n}{n}\right)^m.\]

\end{block}
\end{frame}





\section{Coclusion}

\begin{frame}{Current plan}
Things I have done so far
\begin{itemize}
\item<1-> Understood (mainly symmetric) one dimensional random walk
\item<2-> Defined and proven basic multidimensional theorems
\end{itemize}

\medskip
\visible<4->{
\begin{block}{My goals}
\begin{itemize}
\item Find probabilites of occupation times of given sets in more dimensions
\item Implement in R/Python for simulation study
\item Statistical tests
\end{itemize}
\end{block}}

\end{frame}

\begin{frame}
   \Huge{\centerline{Thank for your attention}}
\end{frame}
\end{document}
