%%% This file contains definitions of various useful macros and environments %%%
%%% Please add more macros here instead of cluttering other files with them. %%%

%%% Minor tweaks of style

% These macros employ a little dirty trick to convince LaTeX to typeset
% chapter headings sanely, without lots of empty space above them.
% Feel free to ignore.
\makeatletter
\def\@makechapterhead#1{
  {\parindent \z@ \raggedright \normalfont
   \Huge\bfseries \thechapter. #1
   \par\nobreak
   \vskip 20\p@
}}
\def\@makeschapterhead#1{
  {\parindent \z@ \raggedright \normalfont
   \Huge\bfseries #1
   \par\nobreak
   \vskip 20\p@
}}
\makeatother

% This macro defines a chapter, which is not numbered, but is included
% in the table of contents.
\def\chapwithtoc#1{
\chapter*{#1}
\addcontentsline{toc}{chapter}{#1}
}

% Draw black "slugs" whenever a line overflows, so that we can spot it easily.
\overfullrule=1mm

\newtheoremstyle{break}
  {\topsep}{\topsep}%
  {\itshape}{}%
  {\bfseries}{}%
  {\newline}{}%


%%% Macros for definitions, theorems, claims, examples, ... (requires amsthm package)
%%%%%%%%%%%%%%%\theoremstyle{plain}
\theoremstyle{break}
\newtheorem{thm}{Theorem}
\newtheorem{lemma}[thm]{Lemma}
\newtheorem{claim}[thm]{Claim}

\theoremstyle{definition}
\newtheorem{defn}{Definition}

\theoremstyle{remark}
\newtheorem*{cor}{Corollary}
\newtheorem*{rem}{Remark}
\newtheorem*{example}{Example}

%%% An environment for proofs

\newenvironment{myproof}{
  \par\medskip\noindent
  \textit{Proof}.
}{
\newline
\rightline{$\qedsymbol$}
}

%%% An environment for typesetting of program code and input/output
%%% of programs. (Requires the fancyvrb package -- fancy verbatim.)

\DefineVerbatimEnvironment{code}{Verbatim}{fontsize=\small, frame=single}

%%% The field of all real and natural numbers
\newcommand{\R}{\mathbb{R}}
\newcommand{\N}{\mathbb{N}}

%%% Useful operators for statistics and probability
\DeclareMathOperator{\pr}{\mathsf{P}}
\DeclareMathOperator{\E}{\mathsf{E}\,}
\DeclareMathOperator{\var}{\mathrm{var}}
\DeclareMathOperator{\sd}{\mathrm{sd}}

%%% Transposition of a vector/matrix
\newcommand{\T}[1]{#1^\top}

%%% Various math goodies
\newcommand{\goto}{\rightarrow}
\newcommand{\gotop}{\stackrel{P}{\longrightarrow}}
\newcommand{\maon}[1]{o(n^{#1})}
\newcommand{\abs}[1]{\left|{#1}\right|}
\newcommand{\dint}{\int_0^\tau\!\!\int_0^\tau}
\newcommand{\isqr}[1]{\frac{1}{\sqrt{#1}}}

%%% Various table goodies
\newcommand{\pulrad}[1]{\raisebox{1.5ex}[0pt]{#1}}
%%%%%%%%%%%%%%%\newcommand{\mc}[1]{\multicolumn{1}{c}{#1}}

%%% Moje vlastní makra

\newcommand{\Z}{\mathbb{Z}}
\newcommand{\pinf}{{+\infty}} %%%Plus nekonečno
\newcommand{\lb}{{\left\{}}
\newcommand{\rb}{{\right\}}}
\newcommand{\dosad}{:=}
\NewDocumentCommand {\suma} { O{i} O{1} O{n} } {\sum\limits_{#1=#2}^{#3}}
\newcommand{\indikator}[1]{\mathbf{1}_{[#1]}}
\newcommand{\cen}{\textbf{check english}}
\newcommand{\rw}{$\left(\{S_n\}_{n=0}^\pinf, p \right)$}
\newcommand{\iid}{independent and identically distributed }
\newcommand{\Iid}{independent and identically distributed }
\newcommand{\Lim}[1]{\raisebox{0.5ex}{\scalebox{0.8}{$\displaystyle \lim_{#1}\;$}}}
\newcommand{\Time}{time }
\newcommand{\intime}{in~time }
\newcommand{\Wlog}{Without loss of generality }
\newcommand{\h}{\frac{1}{2}}
\newcommand{\ce}{\textbf{check english }}
\newcommand{\Mid}{\;\middle\vert\;}
\newcommand{\srw}{$\{S_n\}_{n=0}^\pinf$}
\newcommand{\Lrws}{Let \srw\ be a symmetric random walk}
\newcommand{\Lrw}{Let \rw\ be a random walk}
\newcommand{\Lrwm}{Let $m \in \N$ and \rw be a Type II random walk in $\Z^m$ }
\newcommand\undermat[2]{% http://tex.stackexchange.com/a/102468/5764
  \makebox[0pt][l]{$\smash{\underbrace{\phantom{%
    \begin{matrix}#2\end{matrix}}}_{\text{$#1$}}}$}#2}
\NewDocumentCommand {\indexx} {O{i} O{n} O{x}} {{#3}_{#2}^{#1}}
\NewDocumentCommand {\x} {O{i} O{n}} {x_{#2}^{#1}}
\NewDocumentCommand {\s} {O{i} O{n}} {S_{#2}^{#1}}
\newcommand{\indep}{\perp \!\!\! \perp}
\newcommand{\jaz}[1]{\{1,2,\ldots, #1\}}
\newcommand{\jm}{\jaz{m}}
\newcommand{\jn}{\jaz{n}}
\newcommand{\ltp}{\overset{LTP}{=}}
\newcommand{\gpr}{\overset{GPR}{=}}
\newcommand{\zdef}{\overset{\text{def}}{=}}
\newcommand{\induk}{\overset{IA}{=}}


\newcommand{\todo}[1]{}
\renewcommand{\todo}[1]{{\color{red} TODO: {#1}}}
