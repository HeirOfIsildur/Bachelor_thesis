\chapter{Simple random walk in one dimension}


\begin{defn}[Simple random walk in $\mathbb{Z}$]\label{defn-simple_random_walk_Z}
 Let $\{X_n\}_{n=0}^\pinf$ be a sequence of \iid random variables with values in $\{ -1, +1\}$, that $\forall n \in \N$ satisfy the conditions $\pr \left( X_n=1 \right) =p \in \left( 0,1 \right) $ and $\pr \left( X_n=-1 \right) =1-p=:q$.

 Let $S_0=0$ and $S_n=\suma[i][1][n]X_i$. We call the pair $ \left( \{S_n\}_{n=0}^\pinf, p \right) $ \emph{Simple random walk in $\mathbb{Z}$}.

 In case that $p=q=\h $ we call the pair $ \left( \{S_n\}_{n=0}^\pinf, p \right) $ \emph{Symmetric simple random walk in $\mathbb{Z}$}.
\end{defn}
\begin{rem}
 Very often we refer to $n$ as \Time, $X_i$ as i-th step and $S_n$ as position \intime $n$.
 In simple random walk in $\Z$ we refer to $X_i=+1$ as i-th step was rightwards and to $X_i=-1$ as i-th step was leftwards. If not stated otherwise, we assume that $S_0=0.$
\end{rem}

\begin{defn}[Set of possible positions]\label{defn-set_all_possible_values}
 \Lrw. We call the set $A_n=\{z \in \Z; \abs{z}\leq n, \frac{z+n}{2}\in \Z \}$ \emph{set of all possible positions} of random walk \rw \intime $n$.
\end{defn}


\begin{thm}[Probability of position $x$ \intime $n$]\label{thm-probability_position_time}
 \Lrw and $A_n$ its set of possible positions.
 \[
 \pr \left( S_n=x \right) =
 \begin{cases}
 \binom{n}{\frac{n+x}{2}}p^{\frac{n+x}{2}}q^{\frac{n-x}{2}} & \text{for $x \in A_n$ },\\
 0, & \text{for $x \not\in A_n$}.
 \end{cases}
 \]

\end{thm}

\begin{proof}
 Let us define new variables $r_i=\indikator{X_i=1}, l_i=\indikator{X_i=-1}, R_n=\suma r_i, L_n=\suma l_i$.
 $r_i$ can be interpreted as indicator wether i-th step was rightwards. Then $R_n$ is number of rightwards steps and $L_n$ is number of leftwards steps. We can easily see that $R_n+L_n=n$ and $R_n-L_n=S_n$.
 Therefore we get by adding these two equations $R_n=\frac{S_n+n}{2}.$

 $r_i$ has alternative distribution with parameter $p$ (\textit{Alt(p)}). Therefore $R_n$ as a sum of \iid random variables with distribuion \textit{Alt(p)} has binomial distribution with parameters $n$ and $p$ (\textit{Bi(n,p)}).
 Therefore we get $\pr \left(R_n=x \right) =\binom{n}{x}p^{x}q^{n-x}$. Where we define $\binom{a}{x}\dosad 0$ for $a \in \N, x\in \R \setminus \N$, $x<0$, $x>n$.
 Therefore we get $\pr \left( S_n=x \right) =\pr\left( R_n=\frac{x+n}{2} \right) = \binom{n}{\frac{x+n}{2}}p^{\frac{n+x}{2}}q^{\frac{n-x}{2}}$.
\end{proof}
\begin{lemma}[Spatial homogeneity]\label{lemma-spatial_homogeneity}
  Let $n \in \N, a,b,j \in \Z.$
 $\pr \left( S_n=j \mid S_0=a \right) =\pr \left( S_n=j+b \mid S_0=a+b \right) \forall b \in \Z$
\end{lemma}

\begin{proof}
 $\pr \left( S_n=j \mid S_0=a \right) =\pr \left( \suma X_i=j-a \right)\\
 =\pr \left( \suma X_i= \left( j+b \right) - \left( a+b \right) \right)
 =\pr \left( S_n=j+b \mid S_0=a+b \right).$
\end{proof}


\begin{lemma}[Temporal homogeneity]\label{lemma-temporal_homogeneity}
  Let $n,m \in \N, a,j \in \Z.$
 $\pr \left( S_n=j \mid S_0=a \right) =\pr \left( S_{n+m}=j \mid S_m=a \right) \forall m \in \N$
\end{lemma}
\begin{proof}
 $\pr \left( S_n=j \mid S_0=a \right) =\pr \left( \suma X_i=j-a \right)
 =\pr \left( \suma[i][m+1][m+n]X_i=j-a \right)\\
 =\pr \left( S_{n+m}=j \mid S_m=a \right).$
 Where the second to last equation comes from identical distribution of $\{X_n\}_{n=1}^{\pinf}.$
\end{proof}

\begin{lemma}[Markov property]\label{lemma-markov_property}
 Let $m,n \in \N, n\geq m$ and $a_i \in \Z, i\in \N$. Then $\pr \left( S_n=j \mid S_0=a_0, S_1=a_1, \ldots, S_m=a_m \right)
 =\pr \left( S_n=j \mid S_m=a_m \right) $
\end{lemma}
\begin{proof}
 Once $S_m$ is known, then distribution of $S_n$ depends only on steps ~\\$X_{m+1}, X_{m+2}, \ldots X_n$ and therefore cannot be dependent on any information concerning values $X_1, X_2, \ldots, X_{m-1}$ and accordingly $S_1, S_2, \ldots, S_{m-1}.$
\end{proof}

\begin{rem}\ce-přepsat nějak líp
 In symmetric random walk, everything can be counted by number of possible paths from point to point.
\end{rem}

\begin{defn}[Number of possible paths]\label{defn-number_possible_paths}
 Let $N_n \left( a,b \right) $ be \emph{number of posssible paths} of random walk \rw from point $ \left( 0,a \right) $ to point $ \left( n,b \right)$
 and $N_n^x \left( a,b \right) $ be number of possible paths from point $ \left( 0,a \right) $ to point $\left( n,b \right)$
 that visit point $\left(z,x\right)$ for some $z \in \jn.$
\end{defn}
\begin{thm}\label{thm-number_of_possible_paths}
 Let $a,b \in \Z, n \in \N$ then $N_n \left( a,b \right) =\binom{n}{\h \left(n+b-a \right)}.$
\end{thm}
\begin{proof}
 Let us choose a path from $ \left( 0,a \right) $ to $ \left( n,b \right) $ and let $\alpha$ be number of rightwards steps and $\beta$ be number of leftwards steps.
 Then $\alpha+\beta=n$ and $\alpha-\beta=b-a$. By adding these two equations we get that $\alpha=\h \left( n+b-a \right) $.
 The number of possible paths is the number of ways of picking $\alpha$ rightwards steps from $n$ steps.
 Therefore we get $N_n \left( a,b \right) =\binom{n}{\alpha}=\binom{n}{\h \left( n+b-a \right) }.$
\end{proof}

\begin{thm}[Reflection principle]\label{thm-reflection_principle}
 Let $a,b >0$, then $N_n^0 \left( a,b \right) =N_n \left( -a,b \right) $.
\end{thm}
\begin{proof}
 Each path from $ \left( 0,-a \right) $ to $ \left( n,b \right) $ has to intersect $y=0$-axis at least once at some point.
 Let $k$ be the \Time of earliest intersection with $x$-axis. By reflexing the segment from $\left( 0,-a \right)$ to $\left( k,0 \right)$ in the $x$-axis and letting the segment from $\left(k,0\right)$ to $\left(n,b\right)$ be the same,
 we get a path from point $\left( 0,a \right)$ to $ \left(n,b\right)$ which visits $0$ at point $k$.
 Because reflection is a bijective operation on sets of paths, we get the correspondence between the collections of such paths.
\end{proof}
\begin{defn}[Return to origin]\label{defn-return_origin}
 \Lrw. Let $k \in \N$. We say a \emph{return to origin} occurred \intime $2k$ if $S_{2k}=0$. The probability that \intime $2k$ occured a return to origin shall be denoted by $u_{2k}$.
 We say that \intime $2k$ occurred \emph{first return to origin} if $S_1, S_2, \ldots S_{2k-1}\neq 0$ and $S_{2k}=0$.
 The probability that \intime $2k$ occured first return to origin shall be denoted by $f_{2k}$. By definition $f_0=0$.
 Let $\alpha{2n} \left( 2k \right) $ denote $u_{2k}u_{2 \left( n-k \right) }$
\end{defn}
\begin{thm}[Ballot theorem]\label{thm-ballot_theorem}
 Let $n,b \in N$
 Number of paths from point $ \left( 0,0 \right) $ to point $ \left( n,b \right) $ which do not return to origin is equal to $\frac{b}{n}N_n \left( 0,b \right) $
\end{thm}
\begin{proof}
 Let us call $N$ the number of paths we are refering to.
 Because the path ends at point $ \left( n,b \right) $, the first step has to be rightwards. Therefore we now have $N=N_{n-1} \left( 1,b \right) -N_{n-1}^0 \left( 1,b \right) \overset{\text{T}\ref{thm-reflection_principle}}{=}N_{n-1} \left( 1,b \right) -N_{n-1} \left( -1,b \right).$
 %The last equation was aquired using Reflection principle (\ref{thm-reflection_principle}).

 Therefore we now have:$
 N_{n-1} \left( 1,b \right) -N_{n-1} \left( -1,b \right) =\binom{n-1}{\frac{n}{2}+\frac{b}{2}-1}-\binom{n-1}{\frac{n}{2}+\frac{b}{2}}
 =\frac{ \left( n-1 \right) !}{ \left( \frac{n}{2}+\frac{b}{2}-1 \right) ! \left( \frac{n}{2}-\frac{b}{2} \right) !}-\frac{ \left( n-1 \right) !}{ \left( \frac{n}{2}+\frac{b}{2} \right) ! \left( \frac{n}{2}-\frac{b}{2}-1 \right) !}
 =\frac{ \left( n-1 \right) !}{ \left( \frac{n}{2}+\frac{b}{2}-1 \right) ! \left( \frac{n}{2}-\frac{b}{2} \right) \left( \frac{n}{2}-\frac{b}{2}-1 \right) !}-\frac{ \left( n-1 \right) !}{ \left( \frac{n}{2}+\frac{b}{2} \right) \left( \frac{n}{2}+\frac{b}{2}-1 \right) ! \left( \frac{n}{2}-\frac{b}{2}-1 \right) !}
 =\frac{ \left( n-1 \right) !}{ \left( \frac{n}{2}+\frac{b}{2}-1 \right) ! \left( \frac{n}{2}-\frac{b}{2}-1 \right) !} \left( \frac{1}{\frac{n}{2}-\frac{b}{2}}-\frac{1}{\frac{n}{2}+\frac{b}{2}} \right)
 =\frac{1}{n}\frac{n!}{ \left( \frac{n}{2}+\frac{b}{2}-1 \right) ! \left( \frac{n}{2}-\frac{b}{2}-1 \right) !} \left( \frac{ \left( \frac{n}{2}+\frac{b}{2}-\frac{n}{2}+\frac{b}{2} \right) }{ \left( \frac{n}{2}-\frac{b}{2} \right) \left( \frac{n}{2}+\frac{b}{2} \right)} \right)
 =\frac{b}{n}\frac{n!}{ \left( \frac{n}{2}+\frac{b}{2} \right) ! \left( \frac{n}{2}-\frac{b}{2} \right) !}=\frac{b}{n}\binom{n}{ \left( \frac{n}{2}+\frac{b}{2} \right)}
 =\frac{b}{n}N_n \left( 0,b \right)
 $\end{proof}
 \begin{rem}
   The name \textit{Ballot theorem} comes from the question: In a ballot where candidate $A$ receives $p$ votes and candidate $B$ receives $q$ votes with $p > q$, what is the probability that $A$ will be strictly ahead of B throughout the count?
 \end{rem}
\begin{defn}\label{defn-max}
 $M_n^+=\max \{S_i, i\in \jn\}, M_n^-=\max \{-S_i, i\in \jn\}, M_n^A=\max{M_n^+, M_n^-}$
\end{defn}
\begin{thm}[Probability of maximum up to \Time $n$]\label{thm-probability_maximum_upto_time}
 \Lrw.
 \[\pr \left(  M_n^+\geq r, S_n=b \right) =
 \begin{cases}
 \pr \left( S_n=b \right) & \text{for $b \geq r$ },\\
 \pr \left( S_n=2r-b \right) \left( \frac{q}{p} \right) ^{r-b}, & \text{for $otherwise$}.
 \end{cases}
 \]
\end{thm}
\begin{proof}
 Let us firstly consider the easier case in which $b \geq r$. Because we defined $ M_n^+$ as $\max \{S_i, i\in \jn \}$ we get that $ M_n^+ \geq b \geq r$
 therefore $[ M_n^+ \geq r] \subset [S_n=b]$ therefore we get $\pr \left(  M_n^+\geq r, S_n=b \right) =
 \pr \left( S_n=b \right) $.

 Now let $r\geq 1, b<r$. $N_n^r \left( 0,b \right) $ stands for number of paths from point $ \left( 0,0 \right) $ to point $ \left( n,b \right) $ which reach up to $r$.
 Let $k \in \jn$ denote the first \Time we reach $r$. By reflection principle (\ref{thm-reflection_principle}) , we can reflex the segment from $\left(k,r\right)$ to $ \left( n,b \right) $ in the axis:$y=r$.
 Therefore we now have path from $ \left( 0,0 \right) $ to $ \left( n,2r-b \right) $ and we get that $N_n^r \left( 0,b \right) =N_n \left( 0,2r-b \right) $. $\pr \left( S_n=b, M_n^+\geq r \right) =N_n^r \left( 0,b \right) p^{\frac{n+b}{2}}q^{\frac{n-b}{2}}=
 N_n \left( 0,2r-b \right) p^{\frac{n+ \left( 2r-b \right) }{2}}q^{\frac{n- \left( 2r-b \right) }{2}}p^{\frac{n+b}{2}}q^{\frac{n-b}{2}}p^{b-r}q^{r-b}= \left( \frac{q}{p} \right) ^{r-b}\pr \left( S_n=2r-b \right).$
\end{proof}

\begin{defn}[Walk reaching new maximum at particular \Time]\label{defn-new_maximum}
 Let $b >0$. $f_b \left( n \right) $ denotes the probability that we reach new maximum $b$ \intime $n$. $f_b \left( n \right) =\pr \left( M_{n-1}=S_{n-1}=b-1, S_n=b \right) $
\end{defn}
\begin{thm}[Probability of reaching new maximum $b$ \intime $n$]\label{thm-probability_new_maximum}
 Let $b>0$ then $f_b \left( n \right) =\frac{b}{n}\pr \left( S_n=b \right) .$
\end{thm}
\begin{proof}
 $f_b=\pr \left( M_{n-1}=S_{n-1}=b-1, S_n=b \right)
 =\pr \left(M_{n-1}=S_{n-1}=b-1, X_n=+1\right)
 =p\pr \left( M_{n-1}=S_{n-1}=b-1 \right)\\
 \overset{*}{=}p \left( \pr \left( M_{n-1}\geq b-1, S_{n-1}=b-1 \right) -\pr \left( M_{n-1}\geq b, S_{n-1}=b-1 \right) \right)\\
 \overset{\text{T}\ref{thm-probability_maximum_upto_time}}{=}p \left( \pr \left( S_{n-1}=b-1 \right) -\frac{q}{p}\pr \left( S_{n-1}=b+1 \right) \right)\\
 =p\pr \left( S_{n-1}=b-1 \right) -q \pr \left( S_{n-1}=b+1 \right)\\
 =\binom{n-1}{\frac{n}{2}+\frac{b}{2}-1}p^{\frac{n}{2}+\frac{b}{2}}q^{\frac{n}{2}-\frac{b}{2}}-\binom{n-1}{\frac{n}{2}+\frac{b}{2}}
 p^{\frac{n}{2}+\frac{b}{2}}q^{\frac{n}{2}-\frac{b}{2}}
 =p^{\frac{n}{2}+\frac{b}{2}}q^{\frac{n}{2}-\frac{b}{2}} \left( \frac{ \left( n-1 \right) !}{ \left( \frac{n}{2}+\frac{b}{2}-1 \right) ! \left( \frac{n}{2}-\frac{b}{2} \right) !}-\frac{ \left( n-1 \right) !}{ \left( \frac{n}{2}+\frac{b}{2} \right) ! \left( \frac{n}{2}-\frac{b}{2}-1 \right) !} \right)
 =p^{\frac{n}{2}+\frac{b}{2}}q^{\frac{n}{2}-\frac{b}{2}} \left( \frac{ \left( n-1 \right) !}{ \left( \frac{n}{2}+\frac{b}{2} \right) ! \left( \frac{n}{2}-\frac{b}{2} \right) !} \right) \left( \frac{1}{\frac{n}{2}-\frac{b}{2}}-\frac{1}{\frac{n}{2}+\frac{b}{2}} \right)
 =p^{\frac{n}{2}+\frac{b}{2}}q^{\frac{n}{2}-\frac{b}{2}}\frac{b}{n} \left( \frac{n!}{ \left( \frac{n}{2}+\frac{b}{2} \right) ! \left( \frac{n}{2}-\frac{b}{2} \right) !} \right)\\
 =\frac{b}{n}p^{\frac{n}{2}+\frac{b}{2}}q^{\frac{n}{2}-\frac{b}{2}}\binom{n}{\frac{n}{2}+\frac{b}{2}}
 =\frac{b}{n}\pr \left( S_n=b \right).$
 Where $*$ comes from the fact that the event $[M_{n-1}\geq b-1]$ can be split into two disjoint events:
 $[M_{n-1}\geq b-1]=[M_{n-1}\geq b]\cup [M_{n-1}=b-1].$
 Therefore $\pr\left(M_{n-1}\geq b-1\right)=\pr\left(M_{n-1}\geq b\right)+\pr\left(M_{n-1}=b-1\right).$
 Hence: $\pr\left(M_{n-1}=b-1\right)=\pr\left(M_{n-1}\geq b-1\right)-\pr\left(M_{n-1}\geq b\right).$ The same applies for the probability $\pr\left(M_{n-1}=b-1, S_{n-1}=b-1\right)$
 %druhá rovnost $\pr \left( X_n=1 \right) =p$
\end{proof}

\begin{thm}[XXXMean number of visits to $b$ before returning to origin in symmetric random walk]\label{thm-mean_number_visits}
 \Lrws. Mean number $\mu_b$ of visits of the walk to point $b$ before returning to origin is equal to 1.
\end{thm}
\begin{proof}
 aa
\end{proof}

\begin{lemma}[Binomial identity]\label{lemma-binom_identity}
 Let $n,k \in \N, n>k:\binom{n-1}{k}-\binom{n-1}{k-1}=\frac{n-2k}{n}\binom{n}{k}$
\end{lemma}
\begin{proof}
 $\binom{n-1}{k}-\binom{n-1}{k-1}
 =\frac{ \left( n-1 \right) !}{k! \left( n-k-1 \right) !}-\frac{ \left( n-1 \right) !}{ \left( k-1 \right) ! \left( n-k \right) !}
 =\frac{ \left( n-1 \right) !}{ \left( k-1 \right) ! \left( n-k-1 \right) !} \left( \frac{1}{k}-\frac{1}{n-k} \right)\\
 =\frac{1}{n}\frac{n!}{ \left( k-1 \right) ! \left( n-k-1 \right) !}\frac{n-2k}{k \left( n-k \right) }
 =\frac{n-2k}{n}\frac{n!}{k! \left( n-k \right) !}=\frac{n-2k}{n}\binom{n}{k}$
\end{proof}

\begin{lemma}[Main lemma]\label{lemma-main_lemma}
 Let \rw be a symmetrical random walk.
 Then $\pr \left( S_1, S_2, \ldots, S_{2n}\neq 0 \right) =\pr \left( S_{2n}=0 \right).$
\end{lemma}
\begin{proof}
 $\pr \left(S_1, S_2, \ldots, S_{2n}\neq 0 \right)
 \ltp \suma[i][-\infty][\pinf] \pr \left( S_1 , S_2, \ldots, S_{2n}\neq 0, S_{2n}=2i \right)
 =\suma[i][-n][n] \pr \left( S_1 , S_2, \ldots, S_{2n}\neq 0, S_{2n}=2i \right)
 \overset{*}{=}2\cdot \suma \pr \left( S_1 , S_2, \ldots, S_{2n}\neq 0, S_{2n}=2i \right)
 \overset{\text{T}\ref{thm-ballot_theorem}}{=}2\suma \frac{2i}{2n}\pr \left( S_{2n}=2i \right) =2\suma \frac{2i}{2n}\binom{2n}{n+i}2^{-2n}
 \overset{\text{L}\ref{lemma-binom_identity}} 2\cdot 2^{-2n}\suma \left( \binom{2n-1}{n-i}-\binom{2n-1}{n-i-1} \right) \overset{**}{=}2\cdot 2^{-2n}\binom{2n-1}{n}
 =2^{-2n}\frac{2n}{n}\binom{2n-1}{n}=2^{-2n}\frac{2n \left( 2n-1 \right) !}{m \left( m-1 \right) !m!}
 =2^{-2n}\frac{ \left( 2n \right) !}{m!m!}=2^{-2n}\binom{2n}{n}=\pr \left( S_2n=0 \right).$
 Where $*$ comes from the fact that the random walk is symmetric and $**$ comes from the fact that the positive part of $i$-th term cancels against the negative part of $i+1$-st term.
\end{proof}
\begin{thm}\label{thm-return_origin_upto_time}
 \Lrws. The probability that the last return to origin up to \Time $2n$ occurred \intime $2k$
 is $\pr \left( S_{2k}=0 \right) \pr \left( S_{2 \left( n-k \right) }=0 \right) $.
\end{thm}
\begin{proof}
 $\alpha{2n} \left( 2k \right) =u_{2k}u_{2 \left( n-k \right) }
 =\pr \left( S_{2k}=0 \right) \pr \left( S_{2k+1}, S_{2k+2}, \ldots, S_{2n}\neq 0 \mid S_{2k}=0 \right)
 =\pr \left( S_{2k}=0 \right) \pr \left( S_{1}, S_{2}, \ldots, S_{2 \left( n-k \right) }\neq 0 \right)
 =\pr \left( S_{2k}=0 \right) \pr \left( S_{2 \left( n-k \right) }=0 \right) $
\end{proof}
\begin{thm}\label{thm-probability_position_b_at_time_n_without_return_origin}
 Let $b \in Z$. $\pr \left( S_1, S_2, \ldots, S_n\neq 0, S_n=b \right)
 =\frac{\abs{b}}{n}\pr \left( S_n=b \right) $.
\end{thm}
\begin{proof}
 Let us without loss of generality assume that $b>0$. In that case, first step has to be rightwards $\left(X_1=+1\right).$ Now we have path from point $ \left( 1,1 \right) $ to point $ \left( n,b \right) $ that does not return to origin.
 By Ballot theorem \ref{thm-ballot_theorem} there are $\frac{b}{n}N_n \left( 0,b \right) $ such paths. Each path consists of $\frac{n+b}{2}$ rightwards steps and $\frac{n-b}{2}$ leftwards steps.
 Therefore $\pr \left( S_1\cdot S_2 \cdot, \ldots, S_n\neq 0, S_n=b \right)
 =\frac{b}{n}N_n \left( 0,b \right) p^{\frac{n+b}{2}}q^{\frac{n-b}{2}}
 =\frac{b}{n}\pr \left( S_n=b \right) $. Case $b<0$ is identical.
\end{proof}

\begin{lemma}\label{lemma-probability_strictly_above}
 \Lrws. $\pr \left( S_1, S_2, \ldots, S_{2n}>0 \right)
 =\h \pr \left( S_{2n}=0 \right) =\h u_{2n}.$
\end{lemma}
\begin{proof}
 Because $S_i>0 \forall i \in \N$ the first step has to be rightwards $\left(X_1=S_1=1\right).$ Therefore we get
 $\pr \left( S_1, S_2, \ldots, S_{2n}>0 \right)
 =\suma[r][1][n]\pr \left( S_1, S_2, \ldots, S_{2n}> 0, S_{2n}=2r \right)$.
 The $r$-th term follows equation:
 $\pr \left( S_1, S_2, \ldots, S_{2n}> 0, S_{2n}=2r \right)\\
 =\pr \left( X_1=1, S_2, \ldots, S_{2n}> 0, S_{2n}=2r \right)\\
 =\h \pr \left( S_2, S_3, \ldots, S_{2n}> 0, S_{2n}=2r \Mid S_1=1 \right)\\
 \overset{*}{=}\h \left( \pr \left( S_{2n}=2r\Mid S_1=1 \right) -\pr \left( S_2, S_3, \ldots , S_{2n}=0, S_{2n}=2r \Mid S_1=1\right) \right)\\
 =\h \left( \h ^{2n-1}N_{2n-1} \left( 1,2r \right) -\h ^{2n-1}N_{2n-1}^0 \left( 1,2r \right) \right)\\
 =\h \h ^{2n-1} \left( N_{2n-1} \left( 1,2r \right) -N_{2n-1}^0 \left( 1,2r \right) \right)\\
 \overset{\text{T}\ref{thm-reflection_principle}}{=}\h \h ^{2n-1} \left( N_{2n-1} \left( 1,2r \right) -N_{2n-1} \left( -1,2r \right) \right)\\
 =\h \h ^{2n-1} \left( \binom{2n-1}{n+r-1}-\binom{2n-1}{n+r} \right).$ Where $*$ comes from decomposition: $[S_{2n}=2r]=[S_{2n}=2r,S_1\cdot S_2 \cdot \ldots \cdot S_2n \neq 0]\cup[S_{2n}=2r,S_1\cdot S_2 \cdot \ldots \cdot S_2n=0].$

 Because of the fact that the negative parts of $r$-th terms cancel against the positive parts of
 $\left( r+1 \right) $-st terms and the sume reduces to just $\h \h ^{2n-1}\binom{2n-1}{n}
 =\h \cdot 2 \cdot \h ^{2n}\binom{2n-1}{n}
 =\h \h ^{2n}\frac{2 \left( 2n-1 \right) !}{n! \left( n-1 \right) !}
 =\h \h ^{2n}\frac{ \left( 2n \right) !}{n!n!}=\h \h ^{2n}\binom{2n}{n}
 =\h \pr \left( S_{2n}=0 \right) =\h u_{2n}.$
\end{proof}
\begin{thm}[No return=return]\label{thm-probability_no_return}
 $\pr \left( S_1, S_2, \ldots, S_{2n}\neq 0 \right)
 =\pr \left( S_{2n}=0 \right) =u_{2n}$
\end{thm}
\begin{proof}
 The event $[S_1, S_2, \ldots, S_{2n}\neq 0]$ can be split into two disjoint events:
 $=[S_1, S_2, \ldots, S_{2n} < 0]\cup [S_1, S_2, \ldots, S_{2n} > 0]$.
 By previous theorem (\ref{lemma-probability_strictly_above}) we get that probability
 of both of them is $\h u_{2n}$.
 Because the the events are disjoint we can sum their probabilities and we get the result.
\end{proof}
\begin{lemma}\label{lemma-probability_above_or_on}
 $\pr \left( S_1, S_2, \ldots, S_{2n}\geq0 \right)
 =\pr \left( S_{2n}=0 \right) =u_{2n}$
\end{lemma}
\begin{proof}
 $\h u_{2n}=\pr \left( S_1, S_2, \ldots, S_{2n}>0 \right)
 =\pr \left( X_1=1, S_2, S_3 \ldots, S_{2n}\geq 1 \right)\\
 \nasobeni \pr \left(S_1=1 \right)\pr \left(S_2, S_3 \ldots, S_{2n}\geq 1 \Mid S_1=1\right)\\
 =\h \pr \left( S_2, S_3 \ldots, S_{2n}\geq 1 \mid S_1=1 \right)\\
 \overset{\text{L}\ref{lemma-temporal_homogeneity}}\h \pr \left( S_1, S_2 \ldots, S_{2n-1}\geq 1 \mid S_0=1 \right)\\
 \overset{\text{L}\ref{lemma-spatial_homogeneity}}\h \pr \left( S_1, S_2 \ldots, S_{2n-1}\geq 0 \right)\\
 =\h \pr \left( S_1, S_2 \ldots, S_{2n}\geq 0 \right).$
 Because $[S_{2n-1}\geq 0] \Rightarrow [S_{2n-1}\geq 1] \Rightarrow [S_{2n}\geq 0]$
 Therefore $\pr \left( S_1, S_2 \ldots, S_{2n}\geq 0 \right) =u_{2n}$.
\end{proof}
\begin{thm}\label{thm-f_2n}
 $f_{2n}=u_{2n-2}-u_{2n}$
\end{thm}
\begin{proof}
 The event $[S_1, S_2, \ldots S_{2n-1}\neq 0]$ can be split into two disjoint events:
 $[S_1, S_2, \ldots S_{2n-1}\neq 0, S_{2n}=0]$ and $[S_1, S_2, \ldots S_{2n-1}\neq 0, S_{2n}\neq 0]$.
 Therefore $\pr \left( S_1, S_2, \ldots S_{2n-1}\neq 0 \right)
 =\pr \left( S_1, S_2, \ldots S_{2n-1}\neq 0, S_{2n}=0 \right) +\pr \left( S_1, S_2, \ldots S_{2n-1}\neq 0, S_{2n}\neq 0 \right) $.
 Therefore we get $f_{2n}=\pr \left( S_1, S_2, \ldots S_{2n-1}\neq 0, S_{2n}=0 \right)
 =\pr \left( S_1, S_2, \ldots S_{2n-1}\neq 0 \right) -\pr \left( S_1, S_2, \ldots, S_{2n}\neq 0 \right)$.
 Because $2n-1$ is odd. $\pr \left( S_{2n-1}=0 \right) =0$.
 Therefore the first term is equal to $\pr \left( S_1, S_2, \ldots S_{2n-2}\neq 0 \right)$
 which is by \ref{thm-probability_no_return} equal to $u_{2n-2}$. Second term is by \ref{thm-probability_no_return} equal to $u_{2n}$. Therefore we get the result.
\end{proof}

\begin{lemma}\label{lemma-f_2n=frac}
 $f_{2n}=\frac{1}{2n-1}u_{2n}$
\end{lemma}
\begin{proof}
 $u_{2n-2}=\h^{2n-2}\binom{2n-2}{n-1}=4\cdot \h^{2n} \frac{ \left( 2n-2 \right) !}{ \left( n-1 \right) ! \left( n-1 \right) !}
 =\frac{4n^2}{ \left( 2n \right) \left( 2n-1 \right) }\h^{2n}\binom{2n}{n}
 =\frac{2n}{2n-1}u_{2n}$.
 Therefore $u_{2n-2}-u_{2n}=u_{2n} \left( \frac{2n}{2n-1}-1 \right)
 =u_{2n}\frac{1}{2n-1}$.
\end{proof}
\begin{lemma}[Decomposition of $f_n$]\label{lemma-decomposition_f_n}
  $u_{2n}=\suma[r][1][n] f_{2r}u_{2n-2r}$
\end{lemma}

\begin{proof}
  $u_{2n}
  \overset{\text{D}\ref{defn-return_origin}}{=}\pr \left(S_{2n}=0\right)
  \ltp \suma[r][1][n] \pr \left(S_{2n}=0,S_1, S,2, \ldots, S_{2r-1}\neq 0 S_{2r}=0\right)
  \nasobeni \suma[r][1][n] \pr \left(S_{2n}=0 \Mid S_1, S,2, \ldots, S_{2r-1}\neq 0 S_{2r}=0\right)\pr \left(S_1, S,2, \ldots, S_{2r-1}\neq 0 S_{2r}=0\right)
  =\suma[r][1][n] \pr \left(S_{2n}=0 \Mid S_{2r}=0\right) f_{2r}
  \overset{\text{L}\ref{lemma-temporal_homogeneity}}{=}\suma[r][1][n] u_{2n-2r}f_{2r}.$
\end{proof}
\begin{thm}[Arcsine law for last visits]\label{thm-arcsine_last_visits}
 Let $k,n \in \N, k\leq n$.
 The probability that up to \Time $2n$ the last return to origin occured \intime $2k$ is given by
 $\alpha_{2n} \left( 2k \right) =u_{2n}u_{2 \left( n-k \right) }$.
\end{thm}
\begin{proof}
 The probability involved can be rewritten as:

 $\pr \left( S_{2k+1}, S_{2k+2}, \ldots, S_{2n}\neq 0, S_{2k}=0 \right)\\
 \nasobeni \pr \left( S_{2k+1}, S_{2k+2}, \ldots, S_{2n}\neq 0\mid S_{2k}=0 \right) \pr \left( S_{2k}=0 \right)\\
 \overset{\text{L}\ref{lemma-temporal_homogeneity}}{=}\pr \left( S_{1}, S_{2}, \ldots, S_{2 \left( n-k \right) }\neq 0 \right) \pr \left( S_{2k}=0 \right)\\
 \overset{\text{T}\ref{thm-probability_no_return}}{=}u_{2 \left( n-k \right) }u_{2k}$

\end{proof}
\begin{defn}[Time spend on the positive and negative sides]\label{defn-time_spent_positive_side}
 \Lrw. We say that the walk spent $\tau$ time units of $n$ on the positive side if $\suma \indikator{S_i>0 \lor S_{i-1}>0}=\tau$.
 Let $\beta_{n} \left( \tau \right) $ denote the probability of such an event.
 We say that the walk spent $\zeta$ time units of $n$ on the negative side if
 $\suma\indikator{S_i<0 \lor S_{i-1}<0}=\zeta$.
\end{defn}










\begin{thm}[Arcsine law for sojourn times-OWN PROOF]\label{thm-arcsine_sojourn_times}
 \Lrws. Then $\beta_{2n} \left( 2k \right) =\alpha_{2n} \left( 2k \right) $.
\end{thm}
\begin{proof}%My own proof
 Firstly let us start with degenerate cases. $\beta_{2n} \left( 2n \right)\\
 =\pr \left( \suma[i][1][2n] \indikator{S_i>0 \lor S_{i-1}>0}=2n \right)
 \overset{\text{L}\ref{lemma-probability_above_or_on}}{=}\pr \left( S_1, S_2, \ldots, S_{2n}\geq 0 \right) =*u_{2n}$. By symmetry $\beta_{2n} \left( 0 \right) =\beta_{2n} \left( 2n \right) =u_{2n}.$

 Let $1 \leq k \leq v-1$, where $0\leq v \leq n$. For such $k$ stands equation:
 $
 \beta_{2n} \left( 2k \right) \overset{\text{D}\ref{defn-time_spent_positive_side}}{=} \pr \left( \suma \indikator{S_i>0 \lor S_{i-1}>0}=2k \right)\\
 \ltp \suma[r][1][n]\pr \left( \suma \indikator{S_i>0 \lor S_{i-1}>0}=2k, S_1, S_2, \ldots, S_{2r-1}\neq 0, S_{2r}=0 \right)\\
 \overset{*}{=} \suma[r][1][n]\pr \left( \suma \indikator{S_i>0 \lor S_{i-1}>0}=2k, S_1, S_2, \ldots, S_{2r-1}< 0, S_{2r}=0 \right)\\
 +\suma[r][1][n]\pr \left( \suma \indikator{S_i>0 \lor S_{i-1}>0}=2k, S_1, S_2, \ldots, S_{2r-1}> 0, S_{2r}=0 \right)\\
 \nasobeni \suma[r][1][n]\pr \left( \suma \indikator{S_i>0 \lor S_{i-1}>0}=2k \Mid S_1, S_2, \ldots, S_{2r-1}< 0, S_{2r}=0 \right)\\
 \pr \left( S_1, S_2, \ldots, S_{2r-1}< 0, S_{2r}=0 \right)\\
 +\suma[r][1][n]\pr \left( \suma \indikator{S_i>0 \lor S_{i-1}>0}=2k \Mid S_1, S_2, \ldots, S_{2r-1}> 0, S_{2r}=0 \right)\\
 \pr \left( S_1, S_2, \ldots, S_{2r-1}>0, S_{2r}=0 \right)\\
 \overset{**}{=} \suma[r][1][n] \h f_{2r} \pr \left( \suma[i][2r+1][2n] \indikator{S_i>0 \lor S_{i-1}>0}=2k \Mid S_{2r}=0 \right)\\
 + \suma[r][1][n] \h f_{2r} \pr \left( \suma[i][2r+1][2n] \indikator{S_i>0 \lor S_{i-1}>0}=2k-2r \Mid S_{2r}=0 \right)\\
 \overset{\text{L}\ref{lemma-temporal_homogeneity}}{=} \suma[r][1][n] \h f_{2r} \pr \left( \suma[i][1][2n-2r] \indikator{S_i>0 \lor S_{i-1}>0}=2k\right)+ \suma[r][1][n] \h f_{2r} \pr \left( \suma[i][1][2n-2r] \indikator{S_i>0 \lor S_{i-1}>0}=2k-2r\right)
 =\suma[r][1][n] \h f_{2r} \beta_{2n-2r}\left(2k\right)+\suma[r][1][n] \h f_{2r}\beta_{2n-2r}\left(2k-2r\right).$
 Where $*$ comes from the disjoint decomposition of $[S_1, S_2, \ldots, S_{2r-1}\neq 0]=[S_1, S_2, \ldots, S_{2r-1}> 0]\cup[S_1, S_2, \ldots, S_{2r-1}<0]$ and $**$ comes from using the condition that up to time $2r$ the steps were on the positive/negative sides.

 Now let us proceed by induction. Case for $v=1$ is trivial because it implies degenerous case from *f. Let the statment be true for $v \leq n-1$, then $\suma[r][1][n] \h f_{2r} \beta_{2n-2r}\left(2k\right)+\suma[r][1][n] \h f_{2r}\beta_{2n-2r}\left(2k-2r\right)\\
 \induk \suma[r][1][n] \h f_{2r} \alpha_{2n-2r}\left(2k\right)+\suma[r][1][n] \h f_{2r}\alpha_{2n-2r}\left(2k-2r\right)\\
 \overset{\text{D}\ref{defn-return_origin}}{=} \suma[r][1][n] \h f_{2r} u_{2k}u_{2n-2r-2k}+\suma[r][1][n] \h f_{2r} u_{2k-2r}u_{2n-2k}\\
 = \h u_{2k} \suma[r][1][n] f_{2r} u_{2n-2r-2k}+ \h u_{2n-2k} \suma[r][1][n] f_{2r} u_{2k-2r}u_{2n-2k}\\
 \overset{\text{L}\ref{lemma-decomposition_f_n}} \h u_{2n-2k} u_{2k} +\h u_{2n-2k} u_{2k}=u_{2n-2k} u_{2k}\\
 \overset{\text{D}\ref{defn-return_origin}}{=}\alpha_{2n}\left(2k\right).$
\end{proof}
\begin{defn}[Change of a sign]
 \Lrw. We say that \intime $n$ occurred a change of sign if if $S_{n-1}\cdot S_{n+1}=-1$ in other words if $\left(S_{n-1}=+1 \land S_{n+1}=-1\right) \lor \left(S_{n-1}=-1 \land S_{n+1}=+1\right).$
 We shall denote the probability that up to \Time $n$ occured $r$ changes of sign by $\xi_{r, n}$.
\end{defn}
\begin{thm}[Change of a sign]
 \Lrws. The probability $\xi_{r,2n+1}=2\pr\left( S_{2n+1}=2r+1 \right)$
\end{thm}
\begin{proof}
 Feller
\end{proof}
\section{Problem chapter 9 Feller-není dokončeno ani zkotrolováno}
\begin{defn}[$\delta_n, \varepsilon_n^{r,\pm}$]\label{defn-delta_epsilon}
 \Lrws. $\delta_{n}\left( k\right)$ shall denote $\pr\left(\suma \indikator{S_i>0 \lor S_{i-1}>0}=k, S_n=0 \right)$,
 $ \varepsilon_{n}^r\left(k \right)$ shall denote $\pr\left(\suma \indikator{S_i>0 \lor S_{i-1}>0}=k, S_1, S_2, \ldots, S_{r-1}\neq 0, S_r=0, S_n=0 \right)$,
 $\varepsilon_{n}^{r,+}\left(k \right)$ shall denote $\pr\left(\suma \indikator{S_i>0 \lor S_{i-1}>0}=k, S_1, S_2, \ldots, S_{r-1} > 0, S_r=0, S_n=0 \right)$,
 $\varepsilon_{n}^{r,-}\left(k \right)$ shall denote $\pr\left(\suma \indikator{S_i>0 \lor S_{i-1}>0}=k, S_1, S_2, \ldots, S_{r-1} < 0, S_r=0, S_n=0 \right)$.
\end{defn}
\begin{lemma}[Factorization of $\delta_{2n}\left( 2k\right)$]\label{lemma-factorization_lemma}
 $\delta_{2n}\left( 2k\right)
 =\h \suma[r][1][n]\left( f_{2r} \delta_{2n-2r}\left(2k-2r\right)+f_{2r} \delta_{2n-2r}\left( 2r \right)\right)$.
\end{lemma}
\begin{proof}
 Because $S_{2n}=0$ a return to origin must have happened. Let $2r$ the \Time of first return to origin, where $r \in \{1, 2, \ldots, n\}$. By the law of total probability:
 $\delta_{2n}\left( 2k\right)
 \overset{\text{D}\ref{defn-delta_epsilon}}{=}\pr\left(\suma[i][1][2n] \indikator{S_i>0 \lor S_{i-1}>0}=k, S_{2n}=0 \right)\\
 \ltp \suma[r][1][n]\pr\left(\suma[i][1][2n] \indikator{S_i>0 \lor S_{i-1}>0}=2k, S_1, S_2, \ldots, S_{2r-1}\neq 0, S_{2r}=0, S_{2n}=0 \right)\\
 \overset{\text{D}\ref{defn-delta_epsilon}}{=}\suma[r][1][n] \varepsilon_{2n}^{2k}
 \overset{*}{=} \suma[r][1][n]\pr\left(\suma[i][1][2n] \indikator{S_i>0 \lor S_{i-1}>0}=2k, S_1, S_2, \ldots, S_{2r-1}> 0, S_{2r}=0, S_{2n}=0 \right)\\
 +\suma[r][1][n]\pr\left(\suma[i][1][2n] \indikator{S_i>0 \lor S_{i-1}>0}=2k, S_1, S_2, \ldots, S_{2r-1}< 0, S_{2r}=0, S_{2n}=0\right)\\
 =\suma[r][1][n]\varepsilon_{2n}^{2r,+}\left(2k\right)+\suma[r][1][n]\varepsilon_{2n}^{2r,-}\left(2k\right).$

 Where $*$ comes from the disjoint decomposition $[S_1, S_2, \ldots, S_{2r-1} \neq 0]=[S_1, S_2, \ldots, S_{2r-1} > 0]\cup [S_1, S_2, \ldots, S_{2r-1} < 0]$.

 Now let us calculate
 $\varepsilon_{2n}^{2r,+}\left(2k\right)\\
 =\pr\left(\suma[i][1][2n] \indikator{S_i>0 \lor S_{i-1}>0}=2k, S_1, S_2, \ldots, S_{2r-1}> 0, S_{2r}=0, S_{2n}=0 \right)\\
 \nasobeni \pr\left(\suma[i][1][2n] \indikator{S_i>0 \lor S_{i-1}>0}=2k, S_{2n}=0 \mid S_1, S_2, \ldots, S_{2r-1}> 0, S_{2r}=0 \right)\\
 \pr \left( S_1, S_2, \ldots, S_{2r-1}> 0, S_{2r}=0 \right)\\
 \overset{*}{=} \pr\left(\suma[i][2r+1][2n] \indikator{S_i>0 \lor S_{i-1}>0}=2k-2r, S_{2n}=0 \Mid S_{2r}=0 \right)
 \pr \left( S_1, S_2, \ldots, S_{2r-1}> 0, S_{2r}=0 \right)\\
 \overset{**}{=}\pr\left(\suma[i][2r+1][2n] \indikator{S_i>0 \lor S_{i-1}>0}=2k-2r, S_{2n}=0 \Mid S_{2r}=0 \right) \h f_2r\\
  \overset{\text{L}\ref{lemma-temporal_homogeneity}}{=} \pr\left(\suma[i][1][2n-2r] \indikator{S_i>0 \lor S_{i-1}>0}=2k-2r, S_{2n-2r}=0\right) \h f_{2r}\\
 \overset{\text{D}\ref{defn-delta_epsilon}}{=}\delta_{2n-2r} \left( 2k-2r \right)\h f_{2r}.$

 Where $*$ comes from Lemma (\ref{lemma-markov_property}) and using the condition.

 Where $**$ comes from the fact that $f_{2r}\overset{\text{D}\ref{defn-return_origin}}{=}\pr \left(S_1, S_2, \ldots, S_{2r-1}\neq 0, S_{2r}=0\right)=\pr \left(S_1=1, S_2, \ldots, S_{2r-1}> 0, S_{2r}=0\right)+\pr \left(S_1=-1, S_2, \ldots, S_{2r-1}< 0, S_{2r}=0\right)$ and $\pr \left(S_1=-1, S_2, \ldots, S_{2r-1}< 0, S_{2r}=0\right)=\pr \left(S_1=1, S_2, \ldots, S_{2r-1}> 0, S_{2r}=0\right)$ because of symmetry. Hence $\pr \left(S_1=1, S_2, \ldots, S_{2r-1}> 0, S_{2r}=0\right)=\h f_{2r}.$

 Similarly $\varepsilon_{2n}^{2r,+}\left(2k\right)\\
 =\pr\left(\suma[i][1][2n] \indikator{S_i>0 \lor S_{i-1}>0}=2k, S_1, S_2, \ldots, S_{2r-1}< 0, S_{2r}=0, S_{2n}=0 \right)\\
 = \pr\left(\suma[i][1][2n] \indikator{S_i>0 \lor S_{i-1}>0}=2k, S_{2n}=0 \mid S_1, S_2, \ldots, S_{2r-1}< 0, S_{2r}=0 \right)\\
 \pr \left( S_1, S_2, \ldots, S_{2r-1}< 0, S_{2r}=0 \right)\\
 = \pr\left(\suma[i][2r+1][2n] \indikator{S_i>0 \lor S_{i-1}>0}=2k, S_{2n}=0 \Mid S_{2r}=0 \right)
 \pr \left( S_1, S_2, \ldots, S_{2r-1}< 0, S_{2r}=0 \right)\\
 =\pr\left(\suma[i][2r+1][2n] \indikator{S_i>0 \lor S_{i-1}>0}=2k, S_{2n}=0 \Mid S_{2r}=0 \right) \h f_2r\\
 =\pr\left(\suma[i][1][2n-2r] \indikator{S_i>0 \lor S_{i-1}>0}=2k, S_{2n-2r}=0\right) \h f_{2r}\\
 =\delta_{2n-2r} \left( 2k\right)\h f_{2r}.$

 Therefore $\delta_{2n} \left(2k \right)
 =\frac{1}{2}\suma[r][1][n] f_{2r}\delta_{2n-2r} \left( 2k-2r\right)+\frac{1}{2}\suma[r][1][n] f_{2r}\delta_{2n-2r} \left( 2k \right)\\
 =\h \suma[r][1][n]\left( f_{2r} \delta_{2n-2r}\left(2k-2r\right)+f_{2r} \delta_{2n-2r}\left( 2r \right)\right)$
\end{proof}

\begin{thm}[Equidistributional theorem-ALMOST COMPLETE OWN PROOF]
 \Lrws and $n \in \N$, then
 $\forall k,l \in \{0, 1, 2, \ldots, n \}:\delta_{2n}\left( 2k \right)=\delta_{2n}\left( 2l \right)=\frac{u_{2n}}{n+1}.$
\end{thm}
\begin{proof}
 Let us prove this statement by induction in $n$. In case that $n=1$ we have two options for $k$. Either $k=0$ or $k=1$.
 $\delta_2 \left(0 \right)
 =\pr\left( S_1=-1, S_2=0\right)
 =\h u_{2}
 =\pr\left( S_1=+1, S_2=0\right)
 =\delta_2 \left(2 \right).$

 Let the statment be true for all $l\leq n-1$. In that case $\delta_{2(n-l)}\left( 2k \right)=\frac{u_{2(n-l)}}{n-l+1} \forall k \in \{1,2,\ldots, n-l\}$.
 We want to show that $\delta_{2n}=\frac{u_{2n}}{n+1}$.

 Let us calculate $\delta_{2n}
 \overset{\text{L}\ref{lemma-factorization_lemma}}{=}\h \suma[r][1][n]\left( f_{2r} \delta_{2n-2r}\left(2k-2r\right)+f_{2r} \delta_{2n-2r}\left( 2r \right)\right)\\
 \induk \h \suma[r][1][n]\left( f_{2r} u_{2n-2r}\frac{1}{n-r+1}+f_{2r}u_{2n-2r}\frac{1}{n-r+1}\right)
 =\suma[r][1][n]\frac{f_{2r} u_{2n-2r}}{n-r+1}
 \overset{\text{SNAD TO DOKAZU }\text{L}\ref{lemma-binomial_sum}}{=}\frac{u_{2n}}{n+1}$

\end{proof}
\begin{lemma}[Sum of binomials-POTŘEBUJU DOKÁZAT]\label{lemma-binomial_sum}
  $\suma[r][1][n]\frac{f_{2r} u_{2n-2r}}{n-r+1}=\frac{u_{2n}}{n+1}$
\end{lemma}
\begin{proof}
  $f_{2r} u_{2n-2r}
  \overset{\text{L}\ref{lemma-f_2n=frac}}{=}\frac{1}{2r-1}u_{2r}u_{2n-2r}
  \overset{\text{D}\ref{defn-return_origin}}{=}\frac{1}{2r-1}2^{-2r}\binom{2r}{r}2^{-\left(2n-2r\right)}\binom{2n-2r}{n-r}.$

  Therefore
  $\suma[r][1][n]\frac{f_{2r} u_{2n-2r}}{n-r+1}=\suma[r][1][n]\frac{1}{2r-1}\frac{1}{n-r+1}2^{-2n}\binom{2r}{r}\binom{2n-2r}{n-r}\overset{???}{=}\frac{1}{n+1}2^{-2n}\binom{2n}{n}$
\end{proof}
