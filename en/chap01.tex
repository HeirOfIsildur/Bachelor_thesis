\chapter{Title of the first chapter}

\begin{defn}[Simple random walk in $\mathbb{Z}$]\label{def01:1}
  Let $\{X_n\}_{n=0}^\pinf$ be a sequence of \iid random variables with values in $\{ -1, +1\}$. That $\forall n \in \N$ satisfy $\pr(X_n=1)=p \in (0,1)$ and $\pr(X_n=-1)=1-p=q$. Let $S_0=0$ and $S_n=\suma[i][1][n]X_i$. We call the pair $\left(\{S_n\}_{n=0}^\pinf, p \right)$ \emph{Simple random walk in $\mathbb{Z}$}. In case that $p=q=\frac{1}{2}$ we call the pair $\left(\{S_n\}_{n=0}^\pinf, p \right)$ \emph{Symmetric simple random walk in $\mathbb{Z}$}.
\end{defn}

\begin{rem}
  Very often we refer to $n$ as time, $X_i$ as i-th step and $S_n$ as position in time $n$. In simple random walk in $\Z$ we refer to $X_i=+1$ as i-th step was rightwards. If not stated otherwise, we
\end{rem}

\begin{defn}[Set of possible positions]
  Let \rw be a random walk. We call the set $A_n=\{z \in \Z; \abs{z}\leq n, \frac{z+n}{2}\in \Z \}$ set of all possible positions of random walk \rw in time $n$.
\end{defn}

\begin{thm}[Probability of position $x$ in time $n$]
  Let \rw be a random walk and $A_n$ its set of possible positions.
  \[
  \pr(S_n=x)=
  \begin{cases}
  \binom{n}{\frac{x+n}{2}}p^{\frac{n+x}{2}}q^{\frac{n-x}{2}} & \text{for $x \in A_n$ },\\
  0, & \text{for $x \not\in A_n$}.
  \end{cases}
  \]
\end{thm}
\begin{proof}
  Let us define new variables $r_i=\indikator{X_i=1}, l_i=\indikator{X_i=-1}, R_n=\suma r_i, L_n=\suma l_i$. $r_i$ can be interpreted as indicator wether i-th step was rightwards. Then $R_n$ is number of rightwards steps and $L_n$ is number of leftwards steps. We can easily see that $R_n+L_n=n$ and $R_n-L_n=S_n$. Therefore we get by adding these two equations $R_n=\frac{S_n+n}{2}.$

  $r_i$ has alternative distribution with parameter $p$ (\textit{Alt(p)}). Therefore $R_n$ as a sum of \iid random variables with \textit{Alt(p)} has binomial distribution with parameters $n$ and $p$ (\textit{Bi(n,p)}). Therefore we get $\pr(R_n=x)=\binom{n}{\frac{x+n}{2}}p^{\frac{n+x}{2}}q^{\frac{n-x}{2}}$. Where we define $\binom{a}{x}\dosad 0$ for $a \in \N, x\in \R \setminus \N$, $x<0$, $x>n$. Therefore we get $\pr(S_n=x)=\pr\left( R_n=\frac{x+n}{2}\right)=  \binom{n}{\frac{x+n}{2}}p^{\frac{n+x}{2}}q^{\frac{n-x}{2}}.$
\end{proof}



\begin{lemma}[Spatial homogeneity]
  $\pr(S_n=j|S_0=a)=\pr(S_n=j+b|S_0=a+b) \forall b \in \Z$
\end{lemma}
\begin{proof}
  $\pr(S_n=j|S_0=a)=\pr \left(\suma X_i=j-a\right)=\pr \left(\suma X_i=(j+b)-(a+b)\right)=\pr(S_n=j+b|S_0=a+b).$
\end{proof}

\begin{lemma}[Temporal homogeneity]
  $\pr(S_n=j|S_0=a)=\pr(S_{n+m}=j|S_m=a) \forall m \in \N$
\end{lemma}
\begin{proof}
  $\pr(S_n=j|S_0=a)=\pr\left(\suma X_i=j-a \right)=\pr\left(\suma[i][m+1][m+n]X_i=j-a \right)=\pr(S_{n+m}=j|S_m=a).$
\end{proof}

\begin{lemma}[Markov property]
  Let $n\geq m$ and $a_i \in \Z$. Then $\pr(S_n=j|S_0=a_0, S_1=a_1, \ldots, S_m=a_m)=\pr(S_n=j|S_m=a_m)$
\end{lemma}
\begin{proof}
  Once $S_m$ is known, then distribution of $S_n$ depends only on steps $X_{m+1}, X_{m+2}, \ldots X_n$ and therefore cannot be dependent on any information concerning values $X_1, X_2, \ldots, X_m-1$ and accordingly $S_1, S_2, \ldots, S_m-1.$
\end{proof}

\begin{rem}
  In symmetric random walk, everything can be counted by number of possible paths from point to point.
\end{rem}

\begin{defn}[Number of possible paths]
  Let $N_n(a,b)$ be number of posssible paths of random walk \rw from point $(0,a)$ to point $(n,b)$ and $N_n^x(a,b)$ be number of possible paths from point $(0,a)$ to point $(n,b)$ that visit point $(z,x)$ for some $z \in \{0, \ldots n \}$.
\end{defn}
\begin{thm}\label{thm-num_of_paths}
  Let $a,b \in \Z, n \in \N$ then $N_n(a,b)=\binom{n}{\frac{1}{2}(n+b-a)}.$
\end{thm}
\begin{proof}
  Let us choose a path from $(0,a)$ to $(n,b)$ and let $\alpha$ be number of rightwards steps and $\beta$ be number of leftwards steps. Then $\alpha+\beta=n$ and $\alpha-\beta=b-a$. By adding these two equations we get that $\alpha=\frac{1}{2}(n+b-a)$. The number of possible paths is the number of ways of picking $\alpha$ rightwards steps from $n$ steps. Therefore we get $N_n(a,b)=\binom{n}{\alpha}=\binom{n}{\frac{1}{2}(n+b-a)}.$
\end{proof}
\begin{thm}[Reflection principle]\label{thm-reflection}
  Let $a,b >0$, then $N_n^0(a,b)=N_n(-a,b)$.
\end{thm}
\begin{proof}
  Each path from $(0,-a)$ to $(n,b)$ has to intersect $x$-axis at least once at some point. Let $k$ be the time of earlies intersection with $x$-axis. By reflexing the segment from $(0,-a)$ to $(k,0)$ in the $x$-axis, we get a path from point $(0,a)$ to $(n,b)$ which visits $0$ at point $k$. Because reflection is bijective operation on sets of paths, we get the correspondence between the collections of such paths.
\end{proof}
\begin{defn}[Return to origin]
  Let $S_0=0$ \rw be a random walk. Then if $\exists k \in \N$ such that $S_k=0$ then we say that in $k$-th step occured return to origin.
\end{defn}
\begin{thm}[Ballot theorem]\label{thm-ballot}
  Let $n,b \in N$
  Number of paths from point $(0,0)$ to point $(n,b)$ which do not return to origin is equal to $\frac{b}{n}N_n(0,b)$
\end{thm}
\begin{proof}
  Let us call $N$ the number of paths we are refering to.
  Because the path ends at point $(n,b)$, the first step has to be rightwards. Therefore we now have $N=N_{n-1}(1,b)-N_{n-1}^0(1,b)=N_{n-1}(1,b)-N_{n-1}(-1,b)$. The last equation was aquired using Ballot theorem (\ref{thm-reflection}). We now have:\[
    N_{n-1}(1,b)-N_{n-1}(-1,b)=\binom{n-1}{\frac{n}{2}+\frac{b}{2}-1}-\binom{n-1}{\frac{n}{2}+\frac{b}{2}}=\frac{(n-1)!}{(\frac{n}{2}+\frac{b}{2}-1)!(\frac{n}{2}-\frac{b}{2})!}-\frac{(n-1)!}{(\frac{n}{2}+\frac{b}{2})!(\frac{n}{2}-\frac{b}{2}-1)!}=\frac{(n-1)!}{(\frac{n}{2}+\frac{b}{2}-1)!(\frac{n}{2}-\frac{b}{2})(\frac{n}{2}-\frac{b}{2}-1)!}-\frac{(n-1)!}{(\frac{n}{2}+\frac{b}{2})(\frac{n}{2}+\frac{b}{2}-1)!(\frac{n}{2}-\frac{b}{2}-1)!}=\frac{(n-1)!}{(\frac{n}{2}+\frac{b}{2}-1)!(\frac{n}{2}-\frac{b}{2}-1)!}\left( \frac{1}{\frac{n}{2}-\frac{b}{2}}-\frac{1}{\frac{n}{2}+\frac{b}{2}}\right)=\frac{1}{n}\frac{n!}{(\frac{n}{2}+\frac{b}{2}-1)!(\frac{n}{2}-\frac{b}{2}-1)!}\left( \frac{(\frac{n}{2}+\frac{b}{2}-\frac{n}{2}+\frac{b}{2})}{(\frac{n}{2}-\frac{b}{2})(\frac{n}{2}+\frac{b}{2}})\right)=\frac{b}{n}\frac{n!}{(\frac{n}{2}+\frac{b}{2})!(\frac{n}{2}-\frac{b}{2})!}=\frac{b}{n}\binom{n}{(\frac{n}{2}+\frac{b}{2})}=\frac{b}{n}N_n(0,b)
  \]\end{proof}
\begin{defn}
  $M_n=\max \{S_i, i\in \{0, 1, \ldots, n \} \}$
\end{defn}
\begin{thm}[Probability of maximum in time $n$]\label{thm-probability-of-maximum}
  Let \rw be a random walk.
  \[\pr(M_n\geq r, S_n=b)=
    \begin{cases}
    \pr(S_n=b) & \text{for $b \geq r$ },\\
    \pr(S_n=2r-b)\left( \frac{q}{p}\right)^{r-b}, & \text{for $otherwise$}.
    \end{cases}
    \]
\end{thm}
\begin{proof}
  Let us firstly consider the easier case in which $b \geq r$. Because we defined $M_n$ as $\max \{S_i, i\in \{0, 1, \ldots, n \} \}$ we get that $M_n \geq b \geq r$ therefore $[M_n \geq r] \subset [S_n=b]$ therefore we get $\pr(M_n\geq r, S_n=b)=
  \pr(S_n=b)$.
  Let $r\geq 1, b<r$. $N_n^r(0,b)$ stands for number of paths from point $(0,0)$ to point $(n,b)$ which reach up to $r$. Let $k \in \{ 0, 1, \ldots, n \}$ denote the first time we reach $r$. By reflection principle (\ref{thm-reflection}), we can reflex the segment from $(k,r)$ to $(n,b)$ in the axis:$y=r$. Therefore we now have path from $(0,0)$ to $(n,2r-b)$ and we get that $N_n^r(0,b)=N_n(0,2r-b)$. $\pr(S_n=b,M_n\geq r)=N_n^r(0,b)p^{\frac{n+b}{2}}q^{\frac{n-b}{2}}=N_n(0,2r-b)p^{\frac{n+(2r-b)}{2}}q^{\frac{n-(2r-b)}{2}}p^{\frac{n+b}{2}}q^{\frac{n-b}{2}}p^{b-r}q^{r-b}=\left( \frac{q}{p}\right)^{r-b}\pr(S_n=2r-b).$
\end{proof}

\begin{defn}[Walk rechang new maximum at particular time]
  Let $b >0$. $f_b(n)$ denotes the probability that we reach new maximum $b$ at time $n$. $f_b(n)=\pr(M_{n-1}=S_{n-1}=b-1, S_n=b)$
\end{defn}
\begin{thm}[Probability of reaching new maximum $b$ in time $n$]
  Let $b>0$ then $f_b(n)=\frac{b}{n}\pr(S_n=b).$
\end{thm}
\begin{proof}
  $f_b=\pr(M_{n-1}=S_{n-1}=b-1, S_n=b)=p\pr(M_{n-1}=S_{n-1}=b-1)=p\left(\pr(M_{n-1}\geq b-1, S_{n-1}=b-1)-\pr(M_{n-1}\geq b, S_{n-1}=b-1) \right)=p\left(\pr(S_{n-1}=b-1)-\frac{q}{p}\pr(S_{n-1}=b+1)\right)=p\pr(S_{n-1}=b-1)-q \pr (S_{n-1}=b+1)=\binom{n-1}{\frac{n}{2}+\frac{b}{2}-1}p^{\frac{n}{2}+\frac{b}{2}}q^{\frac{n}{2}-\frac{b}{2}}-\binom{n-1}{\frac{n}{2}+\frac{b}{2}}p^{\frac{n}{2}+\frac{b}{2}}q^{\frac{n}{2}-\frac{b}{2}}=p^{\frac{n}{2}+\frac{b}{2}}q^{\frac{n}{2}-\frac{b}{2}}\left(\frac{(n-1)!}{(\frac{n}{2}+\frac{b}{2}-1)!(\frac{n}{2}-\frac{b}{2})!}-\frac{(n-1)!}{(\frac{n}{2}+\frac{b}{2})!(\frac{n}{2}-\frac{b}{2}-1)!} \right)=p^{\frac{n}{2}+\frac{b}{2}}q^{\frac{n}{2}-\frac{b}{2}}\left(\frac{(n-1)!}{(\frac{n}{2}+\frac{b}{2})!(\frac{n}{2}-\frac{b}{2})!}\right)\left(\frac{1}{\frac{n}{2}-\frac{b}{2}}-\frac{1}{\frac{n}{2}+\frac{b}{2}} \right)=p^{\frac{n}{2}+\frac{b}{2}}q^{\frac{n}{2}-\frac{b}{2}}\frac{b}{n}\left(\frac{n!}{(\frac{n}{2}+\frac{b}{2})!(\frac{n}{2}-\frac{b}{2})!} \right)=\frac{b}{n}p^{\frac{n}{2}+\frac{b}{2}}q^{\frac{n}{2}-\frac{b}{2}}\binom{n}{\frac{n}{2}+\frac{b}{2}}=\frac{b}{n}\pr(S_n=b).$
  %druhá rovnost $\pr(X_n=1)=p$ 4. rovnost \ref{thm-probability-of-maximum}}
\end{proof}
\begin{thm}[XXXMean number of visits to $b$ before returning to origin in symmetric random walk]
  Let \rw be a symmetric random walk. Mean number $\mu_b$ of visits of the walk to point $b$ before returning to origin is equal to 1.
\end{thm}
\begin{proof}
  aa
\end{proof}
\begin{defn}[]
  Let \rw be a random walk and $k \in \N$. Than $u_{2k}=\pr(S_{2k}=0)=\binom{2k}{k}2^{-2k}$ denotes the probability that in time $2k$ the walk returned to origin.

  Let \rw be a random walk and $n,k \in \N, n\geq k$. And $\alpha_{2n}(2k)=u_{2k}u_{2(n-k)}.$
\end{defn}
\begin{lemma}[Approximation of nondecreasing function]\label{thm-nondecreasing}
  Let $x \in R$, $\epsilon, \delta>0$ and $f$ be a nondecreasing function that is defined on $(x-\epsilon, x+\delta)$ and has primitive function on intervals $(x-\epsilon, \epsilon)$ and $(x,x+\delta).$ Then $\frac{1}{\epsilon}\int_limits_{x-\epsilon}^x f(y) dy\leq f(x)\leq \frac{1}{\delta}\int_limits_{x}^{x+\delta} f(y) dy$.
\end{lemma}
\begin{proof}
  $f$ is nondecreasing therefore $\forall z \in (x-\epsilon, x): f(z)\leq f(x)$ and $\frac{1}{\epsilon}\int_limits_{x-\epsilon}^x f(y) dy\leq \frac{1}{\epsilon}\int_limits_{x-\epsilon}^x f(x) dy =f(x)$. Similarly $\forall w \in (x, x+\delta): f(w)\geq f(x)$ therefore $\frac{1}{\delta}\int_limits_{x}^{x+\delta} f(y) dy \geq \frac{1}{\delta}\int_limits_{x}^{x+\delta} f(x) dy=f(x).$
\end{proof}
\begin{lemma}[Stirling's approximation formula]
  $n!\sim \left(\frac{n}{e}\right)^n}\sqrt{2\pi n}$ as $n \to \pinf$ which means that
  $\lim_\limits_{n \to \pinf}\frac{n!}{\left(\frac{n}{e}\right)^n}\sqrt{2\pi n}=1$.
\end{lemma}
\begin{proof}%DK PICK 9.6.7.
  Let $n \in \N$. \label{XXXAAA}$\log(n!)=\log(1)+\log(2)+\ldots +\log(n)$. $\log$ is a nondecreasing function that is defined and has primitive funtion on $\mathbb{R}^+$. Prepositions of \ref{thm-nondecreasing} are met. Choosing $\epsilon=\delta=1$ we get that $\int_limits_{n-1}^n \log(y) dy\leq \log(n)\leq \int_limits_{n}^{n+1} \log(y) dy$ and using \ref{XXXAAA} we get $\int_limits_{0}^n \log(y) dy\leq \log(n!)\leq \int_limits_{1}^{n+1} \log(y) dy$. By calculating integrals (first one is improper however convergent). We get that $n\lon(n)-n\leq \log(n!) \leq (n+1)\log(n+1)-n$
\end{proof}
\begin{thm}[Arc sine law for last visit to the origin]
  Let \rw be a symmetric random walk. The probability that the last visit to origin up to time $2n$ oured at time $2k$ is $\pr(S_{2k}=0)\pr(S_{2(n-k)}=0)$.
\end{thm}
\begin{proof}
  $u_{2k}=\pr(S_{2k}=0)=\binom{2k}{k}2^{-2k}=\frac{(2k)!}{\left( k!\right)^2}2^{-2k}\sim \frac{(2k)^{2k}e^{-2k}\sqrt{4\pi k}}{k^{2k}e^{-2k}2^{2k}2\pi k}=\frac{1}{\sqrt{\pi k}}.$ Therefore we get that $\alpha_{2n}(2k)=u_{2k}u_{2(n-k)}\sim \frac{1}{\pi \sqrt{k(n-k)}}$
\end{proof}


\section{Title of the first subchapter of the first chapter}

\section{Title of the second subchapter of the first chapter}
rem
