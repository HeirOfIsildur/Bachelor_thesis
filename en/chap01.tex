\chapter{Basic definitions}

\begin{rem}
  First, let us properly introduce what a random walk is. After stating some of the basic definitions we will move to the core of this thesis which is to explore properties of occupation of a set times.
\end{rem}

\begin{defn}[Simple random walk in $\mathbb{Z}$]\label{defn-simple_random_walk_Z}
 Let $\{X_n\}_{n=0}^\pinf$ be a sequence of  \iid $\{-1,1\}$-valued random variables, that for some $p\in(0,1)$ and for $n\in \mathbb{N}$ satisfy $\pr \left( X_n=1 \right) =p$ and $\pr \left( X_n=-1 \right) =1-p=:q$.

 Let $S_0=0$ and $S_n=\suma[i][1][n]X_i$. We call the pair $ \left( \{S_n\}_{n=0}^\pinf, p \right) $ \emph{Simple random walk in $\mathbb{Z}$}.

 If $p=q=\h $, the pair $\left( \{S_n\}_{n=0}^\pinf, p \right)$ reduces to the element $\{S_n\}_{n=0}^\pinf$ which is called
\emph{Symmetric simple random walk in $\mathbb{Z}$}.
\end{defn}

\begin{rem}
 Very often we refer to $n$ as \Time, $X_i$ as the i-th step and $S_n$ as position of the walk \intime $n$ or after $n$ steps.
 While referring to simple random walk in $\Z$ we refer to $X_i=+1$ as i-th step was rightwards or more often upwards and to $X_i=-1$ as i-th step was leftwards or downwards. If it is not stated otherwise, we will always assume that $S_0=0.$
\end{rem}

\begin{rem}
  The most important element in random walks is the probability of being in position $x$ \intime $n$.
  In order to calculate such probability we have to firstly define what are even possible positions.

  For example it is impossible for the random walk to be in position $x$ \intime $n$ if $x>n$ simply because there have not been enough steps to make it up to $x$. It is also impossible (given the preposition that $S_0=0$) that after even number of steps the random walk is in odd-numbered position and vice versa. Therefore we define the set of possible positions.
\end{rem}

\begin{defn}[Set of possible positions]\label{defn-set_all_possible_values}
 \Lrw. We call the set $A_n=\{z \in \Z; \abs{z}\leq n, \frac{z+n}{2}\in \Z \}$ \emph{set of all possible positions} of random walk \rw\ \intime $n$.
\end{defn}


\begin{thm}[Probability of position $x$ \intime $n$]\label{thm-probability_position_time}
 \Lrw\ and $A_n$ its set of possible positions.
 \[
 \pr \left( S_n=x \right) =
 \begin{cases}
 \binom{n}{\frac{n+x}{2}}p^{\frac{n+x}{2}}q^{\frac{n-x}{2}} & \text{for $x \in A_n$ },\\
 0, & \text{for $x \not\in A_n$}.
 \end{cases}
 \]

\end{thm}

\begin{rem}
  While having the definition of set of possible positions it is easy to prove the theorem by finding random variable with alternative distribution in each step. By summing them we get a variable with binomial distribution and then we simply modify the result to get desired probability.
\end{rem}

\begin{proof}
  Consider random variables $\indikator{X_i=1}$, and
   $\indikator{X_i=-1}$, and define new random variables
   $R_n=\suma \indikator{X_i=1}, L_n=\suma \indikator{X_i=-1}$.  The
   random variable $\indikator{X_i=1}$ can be interpreted as indicator
   whether i-th step was rightwards. Then, $R_n$ is number of
   rightwards steps and $L_n$ is number of leftwards steps. We can
   easily see that $R_n+L_n=n$ and $R_n-L_n=S_n$.  Therefore we get by
   adding these two equations $R_n=\frac{S_n+n}{2}.$

   Clearly, $\indikator{X_i=1}$ has alternative distribution with
   parameter $p$ (\textit{Alt(p)}). Hence, $R_n$ as a sum of \iid
   random variables with distribution \textit{Alt(p)} has binomial
   distribution with parameters $n$ and $p$ (\textit{Bi(n,p)}).
   Therefore we get
   $\pr \left(R_n=x \right) =\binom{n}{x}p^{x}q^{n-x}$, where we define
  binomial coefficients as statet in preface.
   Finally, for $a \in A_n$ we get
   \[
     \pr \left( S_n=x \right) =\pr\left( R_n=\frac{x+n}{2} \right) =
     \binom{n}{\frac{x+n}{2}}p^{\frac{n+x}{2}}q^{\frac{n-x}{2}}.
   \]
\end{proof}

\begin{rem}
  Following are three simple lemmata that simplify many calculations in the rest of thesis. After proving them we can ask ourselves questions regarding the thesis aims.
\end{rem}

\begin{lemma}[Spatial homogeneity]\label{lemma-spatial_homogeneity}
  \Lrw\ and $n \in \N, a,b,j \in \Z.$ Then for all $b \in \mathbb Z$
\[
  \pr \left( S_n=j \mid S_0=a \right) =\pr \left( S_n=j+b \mid
    S_0=a+b \right)
\]
\end{lemma}

\begin{proof} For any $j, a, b \in \mathbb Z$ holds
  \[
    \begin{split}
      \pr \left( S_n=j \mid S_0=a \right) & =\pr \left( \suma X_i=j-a
      \right)
      =\pr \left( \suma X_i= \left( j+b \right) - \left( a+b \right) \right) \\
      & =\pr \left( S_n=j+b \mid S_0=a+b \right).
    \end{split}
 \]
\end{proof}


\begin{lemma}[Temporal homogeneity]\label{lemma-temporal_homogeneity}
  \Lrw\ and $n,m \in \N, a,j \in \Z.$ Then for all $m \in \mathbb N$
   \[
     \pr \left( S_n=j \mid S_0=a \right) =\pr \left( S_{n+m}=j \mid
       S_m=a \right)
   \]
\end{lemma}
\begin{proof} For any $j, a \in \mathbb Z$ and $m \in \mathbb N$
  \[
    \begin{split}
      \pr \left( S_n=j \mid S_0=a \right) & =\pr \left( \suma X_i=j-a\right)
      =\pr \left( \suma[i][m+1][m+n]X_i=j-a \right)\\
      & =\pr \left( S_{n+m}=j \mid S_m=a \right),
    \end{split}
  \]
where the second equality follows from identical distribution of
$\{X_n\}_{n=1}^{\pinf}.$
\end{proof}

\begin{lemma}[Markov property]\label{lemma-markov_property}
 \Lrw $,n,m \in \N, n\geq m, a_i \in \Z, i\in \N$ such that $\pr \left(S_0=a_0, S_1=a_1, \ldots, S_m=a_m\right)>0$. Then
 \[
     \pr \left( S_n=j \mid S_0=a_0, S_1=a_1, \ldots, S_m=a_m \right)
     =\pr \left( S_n=j \mid S_m=a_m \right)
 \]
\end{lemma}
\begin{proof}
 Because $\{X_n\}_{n=1}^\pinf$ is a sequence of independent variables, once $S_m$ is known, then distribution of $S_n$ depends only on steps ~\\$X_{m+1}, X_{m+2}, \ldots X_n$ and therefore cannot depend on any information concerning values $X_1, X_2, \ldots, X_{m-1}$ and accordingly $S_1, S_2, \ldots, S_{m-1}.$
\end{proof}
