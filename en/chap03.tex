\chapter{Returns to the origin}
\begin{rem}
  Following is quite a tricky identity concerning binomial numbers which may not be obvious to the reader at first sight. Therefore we have decided to state it as a lemma.
\end{rem}
\begin{lemma}[Binomial identity]\label{lemma-binom_identity}
 Let $n,k \in \N, n>k$ then
 \[
 \binom{n-1}{k}-\binom{n-1}{k-1}=\frac{n-2k}{n}\binom{n}{k}
 \]
\end{lemma}
\begin{proof}
  \[
  \begin{split}
    & \binom{n-1}{k}-\binom{n-1}{k-1}
     =\frac{ \left( n-1 \right) !}{k! \left( n-k-1 \right) !}-\frac{ \left( n-1 \right) !}{ \left( k-1 \right) ! \left( n-k \right) !}\\
    & =\frac{ \left( n-1 \right) !}{ \left( k-1 \right) ! \left( n-k-1 \right) !} \left( \frac{1}{k}-\frac{1}{n-k} \right)\\
    & =\frac{1}{n}\frac{n!}{ \left( k-1 \right) ! \left( n-k-1 \right) !}\frac{n-2k}{k \left( n-k \right) }
    =\frac{n-2k}{n}\frac{n!}{k! \left( n-k \right) !}=\frac{n-2k}{n}\binom{n}{k}
     \end{split}
  \]
\end{proof}
\begin{rem}
  Thanks to the previous lemma we are able to prove following theorem. After proving the theorem and two corrolaries stated as lemmata we will be finally able to answer our first question.
\end{rem}
\begin{thm}[Probability of no return up to $n$ is equal to return \intime $n$]\label{thm-no_return=return}
 \Lrws\ , then
 \[
   \pr \left( S_1, S_2, \ldots, S_{2n}\neq 0 \right) =\pr \left( S_{2n}=0 \right).
  \]
\end{thm}
\begin{proof}
  \[
  \begin{split}
    & \pr \left(S_1, S_2, \ldots, S_{2n}\neq 0 \right)
    \ltp \suma[i][-n][n] \pr \left( S_1 , S_2, \ldots, S_{2n}\neq 0, S_{2n}=2i \right)\\
    & =2\suma \pr \left( S_1 , S_2, \ldots, S_{2n}\neq 0, S_{2n}=2i \right)\\
    & \overset{\text{T}\ref{thm-ballot_theorem}}{=}2\suma \frac{2i}{2n}\pr \left( S_{2n}=2i \right) =2\suma \frac{2i}{2n}\binom{2n}{n-i}2^{-2n}\\
    & \overset{\text{L}\ref{lemma-binom_identity}}{=} 2\cdot 2^{-2n}\suma \left( \binom{2n-1}{n-i}-\binom{2n-1}{n-i-1} \right) =2\cdot 2^{-2n}\binom{2n-1}{n}\\
    &=2^{-2n}\frac{2n}{n}\binom{2n-1}{n-1}
  =2^{-2n}\binom{2n}{n}=\pr \left( S_{2n}=0 \right).
  \end{split}
  \]
  The second equation follows from the fact that the random walk is symmetric and the sixth equation follows from the fact that it is a telescopic sum.
\end{proof}



\begin{lemma}[Probability of positive path]\label{lemma-probability_strictly_above}
 \Lrws.
 \[
 \pr \left( S_1, S_2, \ldots, S_{2n}>0 \right)
 =\h \pr \left( S_{2n}=0 \right).
 \]
\end{lemma}
\begin{proof}
 \[
 \pr \left( S_1, S_2, \ldots, S_{2n}>0 \right)
 \ltp \suma[r][1][n]\pr \left( S_1, S_2, \ldots, S_{2n}> 0, S_{2n}=2r \right).
 \]
 The $r$-th term follows equation:
 \[
 \begin{split}
   & \pr \left( S_1, S_2, \ldots, S_{2n}> 0, S_{2n}=2r \right)
   =\pr \left( X_1=1, S_2, \ldots, S_{2n}> 0, S_{2n}=2r \right)\\
   & =\h \pr \left( S_2, S_3, \ldots, S_{2n}> 0, S_{2n}=2r \Mid S_1=1 \right)\\
   & =\h \left( \pr \left( S_{2n}=2r\Mid S_1=1 \right) -\pr \left(S_2\cdot S_3 \cdot \ldots \cdot S_{2n-1}=0, S_{2n}=2r \Mid S_1=1\right) \right)\\
   & =\h \left( 2^{-\left(2n-1\right)}N_{2n-1} \left( 1,2r \right) -2^{-\left(2n-1\right)}N_{2n-1}^0 \left( 1,2r \right) \right)\\
   & =\h 2^{-\left(2n-1\right)} \left( N_{2n-1} \left( 1,2r \right) -N_{2n-1}^0 \left( 1,2r \right) \right) \\
   & \overset{\text{T}\ref{thm-reflection_principle}}{=}\h 2^{-\left(2n-1\right)} \left( N_{2n-1} \left( 1,2r \right) -N_{2n-1} \left( -1,2r \right) \right)\\
   & =\h 2^{-\left(2n-1\right)} \left( \binom{2n-1}{n+r-1}-\binom{2n-1}{n+r} \right)
           \end{split}
 \]
     The first equation comes from the disjoint decomposition:
     \[
     [S_{2n}=2r]=[S_1\cdot S_2 \cdot \ldots \cdot S_{2n-1} \neq 0, S_{2n}=2r]\cup[S_1\cdot S_2 \cdot \ldots \cdot S_{2n-1}=0, S_{2n}=2r].
     \]
 Because of the fact that the negative parts of $r$-th terms cancel against the positive parts of
 $\left( r+1 \right) $-st terms and the sum reduces to
 \[
  \begin{split}
   & \h 2^{-\left(2n-1\right)}\binom{2n-1}{n}
   =\h 2^{-2n}\cdot 2\binom{2n-1}{n}
   =\h 2^{-2n}\frac{2n}{n}\binom{2n-1}{n} \\
   &=\h 2^{-2n}\binom{2n}{n}
   =\h \pr \left( S_{2n}=0 \right).
  \end{split}
 \]
\end{proof}
\begin{comment}
\begin{thm}[Probability of no return and return]\label{thm-no_return=return}
 \Lrws. The following equation holds:
 \[\pr \left( S_1, S_2, \ldots, S_{2n}\neq 0 \right)
 =\pr \left( S_{2n}=0 \right).
 \]
\end{thm}
\begin{proof}
 The event $[S_1, S_2, \ldots, S_{2n}\neq 0]$ is the union of two disjoint events:
 $[S_1, S_2, \ldots, S_{2n}\neq 0]=[S_1, S_2, \ldots, S_{2n} < 0]\cup [S_1, S_2, \ldots, S_{2n} > 0]$.

 By previous theorem (L\ref{thm-no_return=return}) we get that the probability
 of both terms is $\h u_{2n}$.
 Because the the events are disjoint we can sum their probabilities and we get the desired result.
\end{proof}
\end{comment}
\begin{lemma}[Probability of non-negative path]\label{lemma-probability_above_or_on}
  \Lrws, then
 \[
 \pr \left( S_1, S_2, \ldots, S_{2n} \geq 0 \right)
 =\pr \left( S_{2n}=0 \right)
 \]
\end{lemma}
\begin{proof}
\[
  \begin{split}
    & \h \pr\left(S_{2n}=0 \right)\overset{\text{L}\ref{lemma-probability_strictly_above}}{=}\pr \left( S_1, S_2, \ldots, S_{2n}>0 \right)
    =\pr \left( X_1=1, S_2, S_3 \ldots, S_{2n}\geq 1 \right)\\
    & \ltp \pr \left(S_2, S_3 \ldots, S_{2n}\geq 1 \Mid S_1=1\right) \pr \left(X_1=1 \right)\\
    &=\h \pr \left( S_2, S_3 \ldots, S_{2n}\geq 1 \mid S_1=1 \right)\\
    & \overset{\text{L}\ref{lemma-temporal_homogeneity}}{=}\h \pr \left( S_1, S_2 \ldots, S_{2n-1}\geq 1 \mid S_0=1 \right)
    \overset{\text{L}\ref{lemma-spatial_homogeneity}}{=}\h \pr \left( S_1, S_2 \ldots, S_{2n-1}\geq 0 \right)\\
    & =\h \pr \left( S_1, S_2 \ldots, S_{2n}\geq 0 \right).
  \end{split}
 \]
 The last equation comes from the fact that
 \[
 [S_1, S_2 \ldots, S_{2n-1}\geq 0 ]=[S_1, S_2 \ldots, S_{2n-1}\geq 1]=[S_1, S_2 \ldots, S_{2n} \geq 0 ].
 \]
\end{proof}

\begin{thm}[Probability of position $x$ \intime $n$ without returning to origin]\label{thm-probability_position_b_at_time_n_without_return_origin}
 \Lrw\ and $x \in \Z$ then
 \[\pr \left( S_1, S_2, \ldots, S_{n}\neq 0, S_n=x \right)
 =\frac{\abs{x}}{n}\pr \left( S_n=x \right).
 \]
\end{thm}
\begin{proof}
 Let us without loss of generality assume that $x>0$. In that case, first step has to be rightwards $\left(X_1=+1\right).$ Now we have path from point $ \left( 1,1 \right) $ to point $ \left( n,x \right) $ that does not return to origin.
 By Ballot theorem (\ref{thm-ballot_theorem}) there are $\frac{x}{n}N_n \left( 0,x \right) $ such paths. Each path consists of $\frac{n+x}{2}$ rightwards steps and $\frac{n-x}{2}$ leftwards steps.
 Therefore $\pr \left( S_1\cdot S_2 \cdot, \ldots, S_n\neq 0, S_n=x \right)
 =\frac{x}{n}N_n \left( 0,x \right) p^{\frac{n+x}{2}}q^{\frac{n-x}{2}}
 =\frac{x}{n}\pr \left( S_n=x \right) $. Case $x<0$ is identical.
\end{proof}
\begin{comment}
\begin{thm}[Asine law for last return to the origin]\label{thm-return_origin_upto_time}
 \Lrws. The probability that the last return to origin up to \Time $2n$ occurred \intime $2k$
 is
 \[
 \pr \left( S_{2k}=0 \right) \pr \left( S_{2n-2k}=0 \right).
 \]
\end{thm}
\begin{proof}
 \[
  \begin{split}
     & \pr \left( S_{2k+1}, S_{2k+2}, \ldots, S_{2n} \neq 0, S_{2k}=0\right)\\
     & \gpr  \pr \left( S_{2k+1}, S_{2k+2}, \ldots, S_{2n}\neq 0 \mid S_{2k}=0 \right)\pr \left( S_{2k}=0 \right)\\
     & \overset{\text{L}\ref{lemma-temporal_homogeneity}}{=}\pr \left( S_{2k}=0 \right) \pr \left( S_{1}, S_{2}, \ldots, S_{2n-2k}\neq 0 \right)
     \overset{\text{T}\ref{thm-no_return=return}}{=}\pr \left( S_{2k}=0 \right) \pr \left( S_{2n-2k}=0 \right).
  \end{split}
 \]
\end{proof}
\end{comment}

\begin{thm}[First return as difference of returns]\label{thm-f_2n}
  \Lrws, then
 \[
 f_{2n}=u_{2n-2}-u_{2n}
 \]
\end{thm}
\begin{proof}
 The event $[S_1, S_2, \ldots S_{2n-1}\neq 0]$ is the union of two disjoint events:
 \[
 [S_1, S_2, \ldots S_{2n-1}\neq 0, S_{2n}=0] \cup [S_1, S_2, \ldots S_{2n-1}\neq 0, S_{2n}\neq 0].
 \]
 Hence we get
\[
  \begin{split}
    f_{2n} & =\pr \left( S_1, S_2, \ldots S_{2n-1}\neq 0, S_{2n}=0 \right)\\
    & =\pr \left( S_1, S_2, \ldots S_{2n-1}\neq 0 \right) -\pr \left( S_1, S_2, \ldots, S_{2n}\neq 0 \right).
  \end{split}
 \]
 Because $2n-1$ is odd. $\pr \left( S_{2n-1}=0 \right) =0$.
 Therefore the first term is equal to $\pr \left( S_1, S_2, \ldots S_{2n-2}\neq 0 \right)$
 which is by \ref{thm-no_return=return} equal to $u_{2n-2}$. Second term is by \ref{thm-no_return=return} equal to $u_{2n}$. Therefore we get the result.
\end{proof}
